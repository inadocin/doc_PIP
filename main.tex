% -----------------------------------------------------------------------------
% Paquetes y configuracion
% -----------------------------------------------------------------------------
\documentclass[11pt,twoside]{report}
\usepackage[utf8]{inputenc}
\usepackage[spanish]{babel}
\usepackage[a4paper,width=170mm,top=25mm,bottom=25mm,bindingoffset=6mm]{geometry}
\usepackage{fancyhdr}
\usepackage[backend=biber]{biblatex}  % bibliografia
\usepackage{csquotes}
\usepackage{graphicx}

\usepackage{longtable}  % required by gloassaries package?
\usepackage[acronym]{glossaries} % nomenclatura y abreviaciones
% Abreviaciones
\newacronym{ag}{AG}{Algoritmo genético}
\newacronym{mrcvc}{MRCVC}{Motor Rotativo de Combustión a Volumen Constante}
\newacronym{rans}{RANS}{\emph{Reynolds-average Navier-Stokes}}
\newacronym{dta}{DTA}{Diámetro del tubo de admisión}
\newacronym{dte}{DTE}{Diámetro del tubo de escape}
\newacronym{lta}{LTA}{Longitud del tubo de admisión}
\newacronym{lte}{LTE}{Longitud del tubo de escape}
\newacronym{iia}{IIA}{Ángulo de apertura del puerto de admisión}
\newacronym{ifa}{IFA}{Ángulo de cierre del puerto de admisión}
\newacronym{eia}{EIA}{Ángulo de apertura del puerto de escape}
\newacronym{efa}{EFA}{Ángulo de cierre del puerto de escape}
\newacronym{unl}{UNL}{Universidad Nacional del Litoral}
\newacronym{unco}{UNCo}{Universidad Nacional del Comahue}

% Nomenclatura
\newglossaryentry{reynolds}{
    name = $R_e$ ,
    description = Número de Reynolds
}
\newglossaryentry{rend_vol}{
    name = $\eta_v$ ,
    description = Rendimiento volumétrico
}
\newglossaryentry{dens_atmo}{
    name = $\rho_{a,i$ },
    description = Densidad del aire atmosférico
    }
\newglossaryentry{vel}{
    name = $V$ ,
    description = Velocidad
}
\newglossaryentry{ang_ciclo}{
    name = $\theta_c$ ,
    description = Ángulo de ciclo
}
\newglossaryentry{ang_cig}{
    name = $\theta_g$ ,
    description = Ángulo del cigüeñal
}

  % esto esta mejor si lo pongo en cada entrada de texto
\makeglossaries

\graphicspath{ {figuras/} }
\addbibresource{bib.bib}

\pagenumbering{roman}

\fancyhead{}
\fancyhead[RO,LE]{Proyecto Integrador Profesional}
\fancyfoot{}
\fancyfoot[LE,RO]{\thepage}
\fancyfoot[LO,CE]{Capítulo \thechapter}
\fancyfoot[CO,RE]{Nicolás Daniel Barrios}

% -----------------------------------------------------------------------------
% Documento
% -----------------------------------------------------------------------------
\begin{document}

% -- Iniciales
\begin{titlepage}
    \begin{center}

        \vspace*{2cm}
        \huge
        \MakeUppercase{\textbf{Diseño de los sistemas de admisión y escape del
        Motor Rotativo de Combustión a Volumen Constante }}

        \vspace{0.8cm}
        Nicolás Daniel Barrios
        \vfill

        \vspace{0.8cm}
        \includegraphics[width=0.5\textwidth]{logo_unco.jpg}
        \vspace{0.8cm}


        \large
        Facultad de Ingeniería \\
        Universidad Nacional del Comahue \\
        Argentina \\
        2021
    \end{center}
\end{titlepage}

\setcounter{page}{1}
\section*{Agradecimientos}
A mi familia y amigos.

\section*{Resumen}
Aca un pequeño resumen.

\tableofcontents
\listoffigures
\listoftables
\printglossary[type=\acronymtype,title=Abreviaciones]
\printglossary[title=Nomenclatura]

% -- Cuerpo del texto
\pagenumbering{arabic}
\chapter{Introducción}

Este trabajo presenta un procedimiento para optimizar la geometría del sistema
de intercambio de gases del \gls{mrcvc}
\cite{toth}, en particular de la geometría y posición en el cuerpo estatórico
de los puertos de admisión y escape, diámetro y longitud de los conductos
correspondientes.

El MRCVC es un proyecto nacido en la Universidad Nacional del Comahue en el
marco del \emph{Proyecto de Investigación Desarrollo de modelos y herramientas
para la simulación de problemas complejos en ingeniería mediante
fluidodinámica computacional (04/I-251)}. Actualmente se encuentra en etapa
de desarrollo.

La optimización de la geometría se realizó con un conjunto de herramientas, en
primer lugar para obtener las curvas de rendimiento volumétrico se utilizó un
simulador de motores de combustión interna, 

Se utilizó un simulador de motores de combustión interna para simular obtener
las curvas características del motor, curvas que se utilizan para dar un
puntaje a una algoritmo genético para optimizar el conjunto de parámetros que
definen la geometría de los sistemas de intercambio de gases y 

La motivación de este trabajo surge de continuar con el desarrollo del Motor
Rotativo de Combustión a Volumen Constante (MRCVC), en particular
mejorar el prediseño de los sistemas de intercambio de gases sentando las bases
para una futura optimización de los mismos en un motor con requisitos de diseño
establecidos.

\chapter{Antecedentes} \label{cap:antecedentes}

\section{Motores de combustión interna}

% Estas son las ideas que quiero escribir aca

Introducción e importancia de seguir mejorando los diseños de motores, van a
estar funcionando un largo tiempo y hay que cuidar las emisiones etc.

Los motores de combustión interna son máquinas cuyo propósito es convertir la
energía química del combustible en energía mecánica.

Hay que seguir desarrollándolos y mejorándolos porque van a seguir estando
mucho tiempo.

Es importante mejorar la eficiencia de los sistemas de intercambio de gases
porque limitan la producción de potencia del motor

\subsection{Simulación del ciclo}
%

Modelo termodinámico, combustión a volumen constante y a presión constante.
%
Modelo de gases.



\subsection{Sistema de intercambio de gases}
%
El sistema de intercambio de gases cumple la función de extraer los gases
quemados de la cámara de combustión de manera eficiente al final de cada
carrera de expansión, además se encarga de admitir una carga de mezcla frezca a
la cámara de combustión para el próximo ciclo.
%
En un motor de cuatro tiempos, este sistema suele estar compuesto por un filtro
de aire, un tubo que conecta el filtro con el cuerpo de mariposa, el cuerpo de
mariposa, un plenum de admisión y un puerto de admisión, como se ve en la
figura \ref{fig:sist_intercambio}. 
%
La cantidad de aire inducatdo limita la cantidad de combustible que se puede
quemar, por este motivo es imporatnte tener un sistema de admisión eficiente.
%
De misma manera, la cantidad de gases quemados que se pueden extraer luego de
cada ciclo limita la cantidad de masa fresca que puede ingresar a la cámara de
combustión.
%
Otros objetivos de del sistema de intercambio de gases son el de preparar la
mezcla\footnote{En el caso de motores SI que admiten mezclas de aire y
combustible} y brindar un flujo que favorezca el proceso de combustión.

\begin{figure}[h!] \centering
\includegraphics[width=0.5\textwidth]{sistema_intercambio_gases.png}
\caption{Sistema de intercambio de gases esquematizado(buscar una libre o hacer
un esquema propio)} \label{fig:sist_intercambio} \end{figure}

Para la simulación del MRCVC se usa una sistema simplificado, que consta de un
tubo de diámetro $D$ y longitud $L$ tanto para la admisión como para el escape,
como se ilustra en la figura \ref{fig:sist_int_mrcvc}.
%
La apertura y cierre de los puertos es controlada por la posición de los
puertos en el estator.

\subsection{Renidimiento Volumétrico} \label{sec:rend_vol}
%
% Por qué es importante este parámetro.

Los conductos de admision restringen el flujo de aire hacia el motor, para
medira la eficiencia con la que se está admitiendo aire al motor se define el
rendimiento volumétrico $\eta_v$.
%
Se define como la relación entre el caudal volumétrico de aire que ingresa al
sistema de admisión y la velocidad a la que cambia el volumen dentro de un
cilindro.


$$ \eta_v = \frac{2 \dot{m}_a}{\rho_{a,i} V_d N} $$

Donde: $\rho_{a,i}$ es la densidad del aire a la entrada del sistema de
admisión (para motres naturallmente aspirados). También se puede definir
$\eta_v$ como:

$$ \eta_v = \frac{m_a}{\rho_{a,i} V_d} $$

Dónde $m_a$ es la masa inductada al cilindro por ciclo.

Hay varios factores que afectan al rendimiento volumétrico, entre los más
importantes están:
%
\begin{enumerate} \item Efectos cuasiestáticos \item Pérdidas de carga por
    fricción viscosa \item Périddas de carga en los puertos de admision y
    escape \item Transferencia de calor en sistema de admisión \item Reglaje de
    los válvulas/puertos \item Bloqueos de flujo en puertos de admisión y
    escape \item Transferencia de calor en el cilindro \item Sintonía del
    puerto de admisión y escape \item Métodos de sobrecarga \end{enumerate}

Para este trabajo es de interés particular la pérdida de carga en los puertos,
el reglaje y la sintonía de en la admisión y escape.
%
En \cite{lopez13} se demostró que se tiene una mejor \emph{performance} del
motor si se ubican los puertos en el cuerpo central del estator, realizando un
optimzación previa de la geometría mediante un barrido paramétrico de las
variables geométricas que determinan la forma, posición y reglaje de los
puertos.
%
% La ubicación angular de los puertos determina la duración de los procesos de
% admisión y escape, además de modificar la forma y el coeficiente de descarga.

Todos estos parámetros se ajustan o tienen en cuenta con un objetivo de motor
en específico, por lo que fue necesario establecer una curva de renidmiento
volumétrico a la que aspirar para que la simulación numérica del ciclo
termodinámico se pueda acoplar al algoritmo de optimización para evaluar los
motores contra la curva de rendimiento requerida.
%
El criterio de diseño/selección de la curva de $\eta_v$ fué el siguiente:

\begin{itemize} \item que tenga un pico de rendimiento entre 4000 y 6000 rpm.
\item que la curva sea suave \end{itemize}

\subsection{Flujo a través de las válvulas}

\subsection{Estrategias de simulación de motores}
%
El motor será simulado con el simulador de motores de combustión interna
ICEsym, este simulador utiliza modelos 0D para la combustión y 1D para el flujo
de gases a través de los conductos (fuera de la cámara de combustión).

Esta es una herramienta muy útil ya que permite evaluar la \emph{performance}
de un motor a un costo computacional bajo, además la manera en que se
implementó la entrada y salida de datos permite utilizar este simuador como una
"caja negra" de modo que se puede implementar en un script como una función a
la que se le da como entrada un conjunto de parámetros y este devuelve los
resultados de la simulación e un formato que permite la lectura y
transformación en información relevante para el estudio que se está realizando.
%
Fué esta caracteristica del programa la que permitió acoplarlo con un algoritmo
genético para realizar la optimización de la geometría.


Como se mencionó en el apartado \ref{sec:rend_vol}, en trabajos previos se
realizó un pre-diseño de los puertos de admisión y escape.
%
En dicho trabajo se utilizaron coeficientes de descarga estimados y constantes
para simular el flujo en los puertos de admisión y escape.
%
Con el objetivo de modelar con mayor precisión el flujo a través de los puertos
se realizó una modificación al código de ICESym que permite utilizar una
variable adicional para modelar al $C_d$, con lo que se puede representar la
dependencia con la apertura del puerto como con la diferencia de presión
instantánea como $C_d = f(lv, \delta P)$.


\section{Motores rotativos}
%
Algún comentario sobre el funcionamiento en general, ventajas y desventajas de
este tipo de motores.
%
Además de el resurgimiento y las aplicaciones.

\section{Motor rotativo de combustión a volumen constante}
%
\subsection{Historia}
%
Primeros años, lo que significó para la universidad y que es está haciendo
actualmente.

\subsection{Geometría}
%
Algunos esquemas, relaciones más importantes, hay que mostrar lo del volumen

\subsection{Ciclo termodinámico}
%
- Ventajas y desventajas - Modelo de solape de cámaras (tesis de Ezequiel) 
%
- Simulación computacional del ciclo operativo y curvas características de un
motor de combustión interna de avanzada (chiquito)
%
- Diseño óptimo preliminar para los puertos de admisión y escape.

% Heywood 2Ed, pag 376
\chapter{Análisis del problema}


\section{Optimización de la geometría}
\subsection{Simulación computacional del ciclo termodinámico MRCVC}
\subsubsection{ICESym}

ICESym \cite{icesym} es un simulador de motores de combustión interna
desarrollado en conjunto por la UNCo y UNL, utiliza modelos unidimensionales
para simular el flujo en los conductos de admisión y escape y modelos cero
dimensionales para el resto de los componentes.

El simulador incluye un modelo para el solape entre cámaras \cite{lopez16} y
modificaciones particulares al MRCVC, para este trabajo en particular  y con el
fin de obtener una mapa del coeficiente de descarga, se ha agregando la
diferencia de presión entre cámara y puerto como variable de modo que $Cd =
f(L_v, \delta P)$.

La combustión se realiza a volumen constante, por lo que es de esperarse
mayores rendimientos de conversión de combustible en relación a motores en los
que la combustión no se realiza a volumen constante.
% hace falta mostrar que esto es así? poner algún gráfico del heywood o cosas
% por el estilo. puedo citar el heywood?

El indicador que se tomará como referencia para evaluar y comparar diferentes
geometrías es el rendimiento volumétrico ($\eta_v$), este parámetro se define
como:

\begin{equation}
    \eta_v = \frac{m_a}{\rho_{a,i}V_d}
\end{equation}

Dónde:
\begin{description}
    \item[$m_i$] es la masa de mezcla fresca inductada
    \item[$\rho_{a,i}$] es la densidad del aire en el puerto de admisión
    \item[$V_d$] es el volumen desplazado
\end{description}

El rendimiento volumétrico tiene una dependencia compleja de varios factores,
este parámetro es el que da forma a las curvas de \emph{performance} que se
suelen ver en literatura ya que indica la cantidad de mezcla fresca disponible
para la combustión. En en caso de motores de inyección directa (tanto de CI SI)

\subsection{Algoritmo Genético}

\section{Flujometrías}
\subsection{}
\subsection{Coeficiente de descarga ($C_D$)}

Se parte de la ecuación para flujo compresible a través de una restricción.

Para determinar el $C_D$ se debe conocer

\begin{description}
    \item[$p_0$] Presión de estancamiento antes de la restricción.
    \item[$T_0$] Temperatura de estancamiento antes de la restricción.
    \item[$p_T$] Presión estática justo después de la restricción.
    \item[$A_R$] Área de referencia.
    \item[$\dot{m}$] Caudal másico.
    \item[$\gamma$] Cociente de capacidades térmicas del gas.
\end{description}

Los valores de presión y temperatura se obtienen de los datos calculados con
ICESym. De estos valores se calcula el $\gamma$ del gas.

El caudal másico y la velocidad a las entradas y salidas se obtiene con
OpenFOAM.

Para flujo no bloqueado se utiliza:
% \begin{math}
% C_D^{-1} = \frac {A_R p_0} {\dot{m} R T_0^{1/2}}
%            \left( \frac{p_T} {p_0} \right)^{1/\gamma}
%            \left{ \frac{2\gamma}{\gamma-1}
%            \left[1 - \left(\frac{p_T}{p_0}^(\gamma-1/\gamma)\right) \right] \right}^{1/2} 
% \end{math}

En caso del que el flujo esté bloqueado, es decir
$p_T/p_0 \le [2/\gamma+1)]^{\gamma/(\gamma - 1)}$
, la ecuación correspondiente es:

\begin{math}
C_D^{-1} =  \frac {A_R p_0} {\dot{m} (R T_0)^{1/2}}
            \gamma^{1/2}
            \left( \frac{2\gamma}{\gamma+1} \right)^{(\gamma+1)/(2(\gamma-1))}
\end{math}

\subsection{Área de referencia}
El área de referencia utilizada por ICESym es el área de cortina.

$$ A_R = A_C = \pi D_v L_v $$

\subsection{Solape de cámaras}
Tanto al inicio como al cierre del puerto se ve solape de cámaras, por lo que
en estos intervalos angulares hay un valor de $C_D$ para cada cámara.

\section{Criterios adoptados}
\subsection{Geometría}
El puerto se hace recto, igual se podría hacer una entrada más suave.

La altura de la ranura se adopta en 2/3 del alto de la cámara, siendo $h_c=0.0441\ mm$

El eje del puerto se hace perpendicular a una línea que pasa entre el centro
del motor y el la línea media del puerto.


% Capítulo 3 - Metodología:

% En este capítulo describo como es el procedimiento realizado en cada paso de
% la optimización:

% -> optimización algoritmo genético y simulación con icesym, tendría que
%    explicar como funciona icesym y el optimizador
% -> freecad + salome
% -> openfoam

\section{Metodología}

Se simulará un MRCVC de 3 paletas usando el simulador de motores de combustión
interna ICESym \cite{icesym} con el propósito de obtener curvas de rendimiento
volumétrico.
%
Estas serán el principal criterio para evaluar el diseño de los sistemas de
intercambio de gases, que consisten de los puertos y conductos de admisión y
escape.
%
La geometría se optimizará mediante un algoritmo genético, el cual transforma
los datos de rendimiento volumétrico en un puntaje representativo de cada
motor.


La simulación con ICESym requiere del conocimiento previo de los coeficientes
de descarga de los puertos los cuales inicialmente serán estimados para poder
realizar la primer iteración con el simulador.
%
Se utilizará la geometría obtenida en esta primer iteración para crear modelos
en CAD de los puertos de admisión y escape para realizar flujometrías virtuales
y así poder obtener los coeficientes de descarga de cada puerto en distintos
grados de apertura.


\section{ICESym}



\section{Algoritmo genético}

Para realizar la optimización de la geometría flujada se utiliza un algoritmo
genético con una población de $N$ motores, los cuales se representan por una
lista de $k$ números asociados a características geométricas de los sistemas de
intercambio de gases, concreto cada motor se representa por:

\begin{verbatim}
[dta dte lta lte iia ifa eia efa]
\end{verbatim}


\section{CAD}
El modelo 3D de los puertos de admisión, escape y los componentes internos del
motor que afectan el flujo y son relevantes a la flujometría se realizaron con
FreeCAD\cite{freecad} para generar un archivo en formato $.BREP$ y luego
salome\cite{salome} para obtener un archivo .stl "cerrado" en formato ASCII

Dada la cantidad de geometrías posibles y el tiempo que toma el proceso, se
realizaron algunas simplificaciones


\section{Flujometrías}

\subsection{Estudio de convergencia de malla}

El tamaño de la malla debe ser lo suficientemente pequeño para que el resultado
no depende del tamaño de la malla, es decir que el resultado no cambie si se
usan tamaños de celdas menores.
%
Dada la cantidad de posiciones diferentes a ensayar y la gran cantidad de
tiempo que significa realizar el estudio para cada posición, se tomará el
tamaño mínimo de malla como el tamaño que surga de realizar el estudio de
convergencia de malla en dos posiciones del rotor, una posición cerca de la
apertura del puerto y otra cercana al cierre del mismo.
%
Se eligen estas posiciones porque son las que presentan los mayores gradientes
de presión.


El procedimiento a seguir es el siguiente:
% en lugar de una lista numerada podría hacer un algoritmo
\begin{enumerate}
    \item surfaceFeatures
    \item blockMesh
    \item snappyHexMesh
    \item checkMesh
\end{enumerate}


\subsection{Condiciones de contorno}
Para determinar las condiciones de contorno se usan los datos 
\subsubsection{Con solape de cámaras}
\subsubsection{Sin solape de cámaras}

\subsection{Modelos de viscosidad}
\subsubsection{k-epsilon}
\subsubsection{k-omega}

\subsection{Tipos de flujo}
\subsubsection{Flujo compresible}
\subsubsection{Flujo incompresible}

\section{Retroalimentación}


\chapter{Resultados}

En las tablas~\ref{tab:mapa_cd_admision} y~\ref{tab:mapa_cd_escape} se muestran los
resultados de realizar las flujometrías de los puertos de admisión y escape.

\begin{table}
  \centering
    \begin{tabular}{cccc} \toprule
      Caso  & lv        & $\Delta P$    & $C_{D}$   \\ \midrule
      0     & 0.016826  & -100331.39    &  0.213882 \\
      0     & 0.106775  & 5723.72       &  0.489375 \\
      0     & 0.016826  & -263797.72    &  0.011021 \\
      0     & 0.106775  & -3296.18      &  0.803197 \\
      0     & 0.016826  & -652902.78    &  0.011106 \\
      0     & 0.106775  & -9613.29      &  0.815804 \\
      0     & 0.016826  & -513568.73    &  0.011280 \\
      0     & 0.106775  & -3232.97      &  0.813186 \\
      1     & 0.026960  & -116996.12    &  0.375219 \\
      1     & 0.096641  & -3643.9       &  0.878414 \\
      1     & 0.026960  & -237724.11    &  0.018632 \\
      1     & 0.096641  & -6684.11      &  0.867774 \\
      1     &  0.02696  & -496509.46    &  0.111212 \\
      1     &  0.09664  & -18256.20     &  0.805830 \\
      1     & 0.026960  & -237724.11    &  0.022716 \\
      1     & 0.096641  & -6684.11      &  0.862647 \\
      2     & 0.047228  & -49343.47     &  0.541857 \\
      2     & 0.076373  & -5712.86      &  0.918061 \\
      2     & 0.047228  & -109348.67    &  0.487137 \\
      2     & 0.076373  & -17090.38     &  0.914182 \\
      3     & 0.067496  & 13.83         &  0.696967 \\
      3     & 0.071759  & -134.24       &  0.707263 \\
      3     & 0.067496  & -100073.52    &  0.731100 \\
      3     & 0.071759  & -24077.34     &  0.723965 \\
      4     & 0.075750  & -11793.31     &  0.946392 \\
      4     & 0.087764  & -33418.12     &  0.235717 \\
      4     & 0.087764  & -10715.70     &  0.221632 \\
      4     & 0.075750  & -5167.81      &  0.897169 \\
      6     & 0.123601  & -73.94        &  0.878522 \\ \bottomrule
    \end{tabular}
  \caption{Mapa de Cd del puerto de escape} \label{tab:mapa_cd_escape}
\end{table}

\begin{table}
  \centering
  \begin{tabular}{cccc} \toprule
      Caso  & lv        & $\Delta P$    & $C_{D}$   \\ \midrule
      0     & 0.014432  & -6574.97      &  0.206543 \\
      0     & 0.081937  & -87.24        &  0.828822 \\
      0     & 0.014432  & 21856.29      &  0.243975 \\
      0     & 0.081937  & -573.65       &  0.738459 \\
      0     & 0.014432  & -19738.67     &  0.222406 \\
      0     & 0.081937  & 519.60        &  0.487115 \\
      0     & 0.081937  & 1571.95       &  0.587277 \\
      0     & 0.014432  & 18077.97      &  0.256415 \\
      0     & 0.014432  & 2668.61       &  0.247292 \\
      0     & 0.081937  & 0.98          &  0.025970 \\
      2     & 0.062951  & -297.79       &  0.816487 \\
      2     & 0.081937  & 292.92        &  0.466147 \\
      2     & 0.062951  & -7374.88      &  0.980617 \\
      2     & 0.081937  & 4953.85       &  0.541619 \\
      3     & 0.071763  & 4092.13       &  0.501641 \\
      3     & 0.025832  & -3689.81      &  0.289852 \\
      4     & 0.069767  & -789.00       &  0.615690 \\
      4     & 0.069767  & 7869.92       &  0.599348 \\
      4     & 0.005564  & -12539.15     &  0.534555 \\
      4     & 0.005564  & -10091.84     &  0.583979 \\ \bottomrule
    \end{tabular}
  \caption{Mapa de Cd del puerto de Admisión} \label{tab:mapa_cd_admision}
\end{table}


La geometría obtenida luego de realizar la optimización con los mapas de Cd
incorporados a la simulación de ICESym se muestra en la figura \ref{fig:geom_nueva}.
%
Se puede ver que la geometría es similar a la inicial, siendo el puerto de
admisión algo menor en cuanto a diámetro que en el caso inicial.

Como es de esperarse, incorporar estos mapa al modelo del motor tiene un efecto
en el comportamiento del mismo, esto se puede observar principalmente en las
curvas de presión del motor.

Se ha obtenido un prediseño de los sistemas de admisión y escape que buscó
maximizar el rendimiento volumétrico y reducir la fracción de gases residuales
en un rango medio a medio alto de revoluciones del motor, obteniendo una curva
de rendimiento volumétrico con una sintonía a 4000 RPM y manteniendo valores de
$\eta_v$ cercanos al 70\% para 6000 a 7000 RPM.
%
Además, se obtuvo un mapa de $C_D$ que modeliza el funcionamiento de los
puertos con apertura de puerto y diferencia de presión como variables.

Junto con estos resultados se desarrolló un conjunto de \emph{scripts} que
permiten utilizar el simulador ICESym como una \emph{caja negra}, pudiendo
conFigurar, ejecutar y leer los resultados de una simulación, permitiendo
utilizar el simulador acoplado a otro programa.

% Como trabajos a futuro se tiene la incorporación de los modelos de fricción
% obtenidos en trabajos anteriores~\parencite{roldan} a ICESym, la creación de un
% optimizador genético híbrido, es decir, que permita utilizar algoritmos
% genéticos para determinar los puntos más interesantes del dominio evaluado para
% un problema dado, y el uso de técnicas directas para encontrar el valor del
% óptimo buscado.

Como posible trabajo a futuro se puede continuar con el proceso de optimización
que se inició en este trabajo, realizando más iteraciones para refinar la
geometría obtenida.
%
Así como también explorar otras funciones objetivo para la optimización con el
algoritmo genético.


% -- Finales
\printbibliography
\include{partes/apendices}
\end{document}

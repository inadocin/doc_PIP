Este trabajo tiene la finalidad de obtener un diseño preliminar de la geometría
de los sistemas de intercambio de gases del Motor Rotativo de Combustión a
Volumen Constante (MRCVC,~\parencite{toth}), con el objetivo general de
maximizar la eficiencia del sistema en un rango de velocidades del motor.
%

El MRCVC es un proyecto que surgió en la Universidad Nacional del Comahue,
presentado por el Ing. Jorge A. Toth en el año 1996 al Instituto Nacional de
la Propiedad Industrial y patentado en el año 1999.

\begin{figure}
    \centering
    \includegraphics[width=0.5\textwidth]{perspectiva_mrcvc.png}
    \caption{Motor Rotativo de Combustión a Volumen Constante}\label{fig:mrcvc}
\end{figure}

En trabajos anteriores~\parencite{lopez13,lopez16,toth00}~se han mencionado
las características que hacen al MRCVC un motor atractivo: la geometría de la
cámara de combustión y del conjunto rotante permiten que gran parte del proceso
de combustión se realice a volumen constante, además de tener un balanceo
mecánico de fuerzas que le permite alcanzar altas velocidades de rotación.
%
Esto promete un funcionamiento más suave del motor, además de una reducción del
ruido y desgaste en comparación a motores rotativos tradicionales (Wankel) y
reciprocantes.
%
Por otro lado, hay que mencionar que los motores rotativos traen consigo una
serie de problemas como la necesidad de introducir aceite a la cámara de
combustión para lubricar elementos móviles, el solape de cámaras durante la
apertura de los puertos y, en particular al MRCVC, un complejo sistema de
sellos~\parencite{roldan20}.

%%%%%%%%%%%%%%%%%%%%%%%%%%%%%%%%%%%%%%%%%%%%%%%%%%%

La motivación de este trabajo surge del deseo de continuar con el desarrollo del
MRCVC y mejorar el pre-diseño de los sistemas de intercambio de gases, sentando
la base para una futura optimización de los mismos en un motor con requisitos de
diseño concretos.


Se buscó obtener un pre-diseño satisfactorio del sistema poniendo énfasis en la
geometría de los puertos de admisión y escape, definiendo las métricas a
utilizar para medir la eficiencia del sistema y poder realizar comparaciones
cuantitativas de los diseños propuestos.
%
Debido al costo computacional de las simulaciones necesarias para realizar esta
optimización se restringe el modelado de la geometría a las definiciones de las
posiciones de los puertos en el estator, largos y diámetros de los conductos.
%
No se repara en detalles como la forma de la transición entre las paredes del
puerto hacia la cámara, el ángulo al estator o detalles similares.

%%%%%%%%%%%%%%%%%%%%%%%%%%%%%%%%%%%%%%%%%%%%%%%%%%%
Se utilizó una serie de herramientas de simulación para la optimización, algunas
de las cuales fueron:

\begin{enumerate}[label=\Alph*., leftmargin=*, noitemsep]
        %
    \item ICESym~\parencite{icesym}, simulador de motores de combustión interna basado en modelos cero-/uni-dimensionales (0D/1D).
        %
    \item OpenFOAM~\parencite{openfoam}, una herramienta libre de CFD (\textit{Computational Fluid Dynamics}).
        %
    \item Salome~\parencite{salome}, plataforma libre para simulación numérica.
        %
\end{enumerate}

\nomenclature[F]{\(CFD\)}{\textit{Computational Fluid Dynamics.}}

Se desarrolló un optimizador capaz de generar y evaluar diferentes geometrías
con el fin de buscar una combinación de parámetros que maximicen indicadores de
eficiencia del sistema, como por ejemplo, el rendimiento volumétrico del motor
para un rango de velocidades determinado.

El proceso de optimización consta de una primera aproximación utilizando como
punto de partida los resultados de trabajos anteriores~\parencite{lopez13}, en
los cuales se evaluó el funcionamiento de los parámetros que definen la
geometría de los sistemas de intercambio de gases, analizándose en particular:
diámetros y longitudes de conductos y reglaje o posición angular de los puertos.

La optimización se realiza con un algoritmo evolutivo (o genético) funcionando
en conjunto con ICESym, el cual provee el puntaje a cada configuración del
motor necesario para estos procesos de optimización.
%
El puntaje se introduce en la función objetivo, la cual evalúa a cada uno de los
candidatos generados por el algoritmo.

El diseño preliminar de la primera ronda de optimización se volcó en un modelo
tridimensional (3D) de los puertos, parametrizado de modo tal que se puede
alterar rápidamente la geometría, modificando variables como el diámetro de los
conductos y la posición relativa en la periferia del motor.
%
Este modelo 3D se utilizó para extraer la geometría a simular con OpenFOAM y
realizar flujometrías de las que se obtiene un valor del flujo másico
($\dot{m}$) en estado estacionario para un punto operativo del motor, es decir,
para una combinación de diferencia de presión entre puerto y cámara
($\Delta P$) y el grado de apertura del puerto ($l_{v}$).
%
El flujo másico se utilizó para medir la eficiencia con la cual escurre el gas a
través del puerto, con el objetivo de crear un mapa del coeficiente de descarga
($C_{D}$) que sea función de las variables mencionadas.
%
Este mapa se utiliza como retroalimentación del simulador de motores ICESym,
para tener un mejor modelado del flujo de gas a través de los puertos en un
rango operativo del motor y con esto realizar una nueva corrida de optimización
a fin de refinar el diseño obtenido en la primera iteración.
%
\nomenclature[PO]{\(\dot{m}\)}{Caudal másico}
\nomenclature[F]{\(C_{D}\)}{Coeficiente de descarga}

% Primer capitulo
%
A continuación se describe la organización del presente trabajo.
%
% Segundo capitulo
%
En el segundo capítulo se presenta una breve descripción del funcionamiento de
los motores de combustión interna, seguido de los indicadores utilizados para
medir el rendimiento de motores en general e indicadores particulares de la
eficiencia de los sistemas de intercambio de gases, como el rendimiento
volumétrico y la fracción de gases residuales.
%
Luego, se describe el funcionamiento del MRCVC, indicando los aspectos
sobresalientes de este motor, además de desventajas del mismo y las posibles
aplicaciones.
%
También se describe el proceso de intercambio de gases y se define el
coeficiente de descarga $C_{D}$, junto con las ecuaciones asociadas.
%
% Tercer capitulo
%
En el tercer capítulo se describen las herramientas computacionales utilizadas
en este trabajo.
%
Se presenta el simulador de motores ICESym, el optimizador desarrollado y
la integración entre ambos programas.
%
Se incluye una descripción del funcionamiento del optimizador, los motivos de
seleccionar un algoritmo de tipo evolutivo o genético, las ventajas y
desventajas, los componentes básicos y finalmente la implementación del mismo.
%
En este capítulo también se presenta el software utilizado para realizar las
flujometrías, OpenFOAM, la implementación de las condiciones iniciales y
de contorno, extracción de datos de ICESym y otras herramientas necesarias para
generar el modelo de CAD del puerto, malla y otros detalles relativos al proceso
de utilizar el programa.

% Cuarto capítulo
%
En el cuarto capítulo se presentan detalles particulares de las simulaciones
realizadas, incluyendo la geometría del motor utilizado y su implementación en
ICESym.
%
Además, se detallan la configuración de las flujometrías virtuales realizadas
con OpenFOAM, condiciones iniciales, configuración de la herramienta, esquemas
de discretización y otros parámetros importantes relacionados a las
flujometrías.
%
% Quinto capítulo
%
En el quinto capítulo se presentan los resultados del trabajo para cada una de
las etapas correspondientes.
%
% Sexto capítulo
%
Por último, se exponen las conclusiones del trabajo, opiniones finales y una
perspectiva a futuro de posibles trabajos a seguir.

\newpage \thispagestyle{plain}

\begin{center} \large{\textbf{{Diseño de los sistemas de admisión y escape del
Motor Rotativo de Combustión a Volumen Constante}}}\\
\end{center} % \setstretch{1.5}

\normalsize \ \hfill Autor: Nicolás Daniel Barrios

\hfill Director: Dr. Ing. Ezequiel José López


\textbf{Resumen}

En este trabajo se optimizaron los sistemas de admisión y escape del Motor
Rotativo de Combustión a Volumen Constante (MRCVC) utilizando herramientas de
simulación computacional tales como ICESym (simulador 0D/1D de motores de
combustión interna), OpenFOAM (herramienta CFD, por\emph{Computer Fluid
Dynamics} de código abierto) y un optimizador basado en un algoritmo genético
(AG) desarrollado con la librería DEAP (Python), entre otros.

Primero, se desarrolló una librería de funciones para acoplar el AG con ICESym,
permitiendo configurar, ejecutar y procesar datos del simulador.
%
También se modificó ICESym, agregando un modelo de coeficientes de descarga
($C_{D}$) dependiente de la diferencia de presión y grado de apertura del
puerto, permitiendo un mejor modelado del flujo de gas a través de los puertos.

Se realizó una optimización inicial de la geometría de los puertos del MRCVC con
valores de $C_{D}$ constantes, buscando maximizar el rendimiento volumétrico.
%
La geometría resultante se modeló con un programa de diseño asistido por
computadora (CAD por sus siglas en inglés) de código abierto, FreeCAD.
%
Este resultado, junto con el estado termodinámico del gas obtenido de los datos
de salida de ICESym, se utilizó para realizar flujometrías virtuales de los
puertos en diferentes configuraciones empleando OpenFOAM, y así obtener el
correspondiente mapa de $C_{D}$.
%
Este mapa se utilizó como retroalimentación del AG para una nueva optimización,
logrando una geometría de puertos satisfactoria para el estado actual desarrollo
del motor.

\noindent

\textit{Palabras clave: MRCVC, Rendimiento volumétrico, Sistemas de intercambio
de gases, CFD, Optimización, Algoritmo Genético.}

\newpage \thispagestyle{plain} % \setstretch{1.0}

\begin{center} \large{\textbf{{Design of the Intake and Exhaust Systems of the
Constant Volume Combustion Rotary Engine}}}\
\end{center} % \setstretch{1.5}

\normalsize \hfill Author: Nicolás Daniel Barrios

\hfill Advisor: Dr. Ing. Ezequiel José López

\textbf{Abstract}

In this work, the intake and exhaust systems of the Constant Volume Combustion
Rotary Engine were optimized using computational simulation tools such
as ICESym (0D/1D internal combustion engine simulator), OpenFOAM (open-source
CFD tool), and a genetic algorithm (GA) based optimizer developed with the DEAP
(Python) library, among others.

First, a library of functions was developed to couple the GA with ICESym,
allowing configuration, execution, and data processing from the simulator.
%
ICESym was also modified to include a discharge coefficient ($C_{D}$) model
dependent on pressure difference and port opening fraction, enabling better
modeling of gas flow through the ports.

An initial optimization of the engine port geometry was performed with constant
$C_{D}$ values, aiming to maximize volumetric efficiency.
%
The resulting geometry was modeled with an open-source computer-aided design
(CAD) program, FreeCAD.
%
This result, along with the gas thermodynamic state obtained from ICESym output
data, was used to conduct virtual flow measurements of the ports in different
configurations using OpenFOAM, thereby obtaining the corresponding $C_{D}$ map.
%
This map was used as feedback for the GA in a new optimization, achieving a
satisfactory port geometry for the current state of engine development.

\noindent

\textit{Keywords: MRCVC, Volumetric Efficiency, Gas Exchange Systems, CFD,
Optimization, Genetic Algorithm.}


\newpage

\newpage
\thispagestyle{plain}

\begin{center}
\large{\textbf{{Diseño de los sistemas de admisión y escape del Motor Rotativo de Combustión a Volumen Constante}}}\\
\end{center}
% \setstretch{1.5}

\normalsize
\
 \hfill Autor: Nicolás Daniel Barrios

\hfill Director: Dr. Ing. Ezequiel José López


\textbf{Resumen}


En este trabajo se realizó una optimización de los sistemas de admisión y escape
del Motor Rotativo de Combustión a Volumen Constante (MRCVC), utilizando
herramientas de simulación computacional como ICESym, un simulador de motores de
combustión interna con modelos cero-/uni-dimensionales (0D/1D);
OpenFOAM, la herramienta libre de fluidodinámica computacional (CFD por sus
siglas en inglés); un optimizador basado en un algoritmo genético (AG)
desarrollado a base de la librería DEAP (Python) para funcionar en conjunto con
ICESym; entre otros.

Como primera instancia se desarrolló una librería de funciones que acoplaron al
AG con ICESym, permitiendo: conFigurar, ejecutar y procesar los datos de salida
del simulador de motores.
%
% Además, se definió una curva objetivo de rendimiento volumétrico que se utilizó
% como único parámetro para evaluar la eficiencia de los sistemas de intercambio
% de gases de los candidatos generados por el AG.

También se realizaron algunas modificaciones a ICESym, siendo la más importante
el agregado de un modelo de coeficientes de descarga ($C_D$) dependiente de dos
variables: presión y apertura de puerto.
%
Esto permitió agregar como dato de entrada un mapa de $C_D$ para tener un mejor
modelado del funcionamiento de los puertos.


Se realizó una primera optimización de la geometría de los puertos del MRCVC con
valores de $C_D$ asumidos constantes, obteniéndose un diseño preliminar de los
puertos cuya geometŕia fue modelada en un programa de diseño asistido por
computadora (CAD por sus siglas en inglés) con el programa FreeCAD.
%
Este resultado, junto con el estado termodinámico del gas obtenido de los datos
de salida de ICESym, se utilizó para realizar flujometrías con OpenFOAM de los
puertos en diferentes regímenes de funcionamiento del motor y así obtener el
mapa de $C_D$.
%
Este mapa se utilizó como retroalimentación del AG para una nueva optimización
de los sistemas de intercambio de gases, obteniendo como resultado una
geometría de los puertos que se considera satisfactoria para el estado actual
de desarrollo del motor.


\noindent 

\textit{Palabras clave: MRCVC, Rendimiento volumétrico, Sistemas de
intercambio de gases, CFD, Optimización, Algoritmo Genético.}

\newpage
\thispagestyle{plain}
% \setstretch{1.0}
\begin{center}
\large{\textbf{{Intake and Exhaust gas exchange systems design for the Constant-Volume Combustion Rotary Engine}}}\\
\end{center}
% \setstretch{1.5}

\normalsize{

\hfill Author: Nicolás Daniel Barrios

\hfill Advisor: Dr. Ing. Ezequiel José López

\textbf{Summary}

This document presents the optimization of the Constant-Volume Combustion
Rotary Engine gas exchange systems.
%
This was achieved using free and open source tools such as ICESym, an internal
combustion engine symulator based on 0D/1D models; OpenFOAM, the open source CFD
tool; a genetic optimization algorithm (GA) tailored to work with ICESym, based
on the DEAP Python library, among others.

A first step was developing a Python library to allow the GA to configure and
run and post-process ICESym simulations.
%
Also, an objective function was defined to evaluate individual and specific
engine conFiguration performance.

To allow the GA to work with ICESym, some modifications to the source code were
made for easier communication with the optimization functions.
%
A better modeling of the gas exchange process was made by adding functionality
to ICESym that takes into account a discharge coefficient map dependant on the
pressure diferential across the port and the port opening area.
%
Using constant value discharge coefficients for the intake and exahust ports, a
first optimization run was performed that resulted in a pre-optimized geometry that
was modeled with FreeCAD.

This 3D CAD model of the port geometry was used with gas state data taken from
ICESym results to make CFD runs. With this results the mass flow rate was
obtained in a variety of engine position and states to build up a discharge
coefficient map. This map was used as an ICESym input to have a better flow
model of the gas through the ports.

Lastly this map was used in a second optimization run to obtain the final
geometry.

\textit{Keywords:MRCVC, Volumetric efficiency, Gas exchange systems, CFD,
Optimization, Genetic algorithm}

\newpage

% \lhead{}
% \rhead{\thepage}


\graphicspath{ {imagenes/} }

% ----------------------------------------------------------------------------
% Polylgossia
\setdefaultlanguage{spanish}
\addto\captionsspanish{\renewcommand{\tablename}{Tabla}}
\addto\captionsspanish{\renewcommand{\listtablename}{Índice de tablas}}

% ----------------------------------------------------------------------------
% fonts
% \setmainfont{Times New Roman}

% ----------------------------------------------------------------------------
\setcounter{secnumdepth}{3}

% ----------------------------------------------------------------------------
% Nomenclature
% DE: https://www.overleaf.com/learn/latex/Nomenclatures
\renewcommand{\nomname}{Nomenclatura}
\renewcommand\nomgroup[1]{%
  \item[\bfseries
  \ifstrequal{#1}{G}{Geométricas}{%
  \ifstrequal{#1}{PO}{Parámetros Operativos}{%
  \ifstrequal{#1}{AG}{Algoritmo Genético}{%
  \ifstrequal{#1}{F}{Flujometrías}{}}}}%
]}

% Geometry
\geometry{a4paper, left=35mm, right=25mm,top=35mm,bottom=25mm,headsep=20mm}

% Luacode
\newcommand\lua[1]{\luaexec{#1}}


% ----------------------------------------------------------------------------
% Listings
% DE: https://www.overleaf.com/learn/latex/code_listing
\definecolor{codegreen}{rgb}{0,0.6,0}
\definecolor{codegray}{rgb}{0.5,0.5,0.5}
\definecolor{codepurple}{rgb}{0.58,0,0.82}
\definecolor{backcolour}{rgb}{0.95,0.95,0.92}
\usepackage{listings}
\lstdefinestyle{mystyle}{
    backgroundcolor=\color{backcolour},
    commentstyle=\color{codegreen},
    keywordstyle=\color{magenta},
    numberstyle=\tiny\color{codegray},
    stringstyle=\color{codepurple},
    basicstyle=\ttfamily\footnotesize,
    breakatwhitespace=false,
    breaklines=true,
    captionpos=b,
    keepspaces=true,
    numbers=left,
    numbersep=5pt,
    showspaces=false,
    showstringspaces=false,
    showtabs=false,
    tabsize=2
}
\lstset{style=mystyle}

% ----------------------------------------------------------------------------
% Bibliografia - biblatex
\addbibresource{otros/bibliografia.bib}

% esta línea añade coma entre el autor y año de la referencia citada
\renewcommand*{\nameyeardelim}{\addcomma\space}

% ----------------------------------------------------------------------------
% hyperref
\hypersetup{
     colorlinks=true,
     linkcolor=blue,
     filecolor=blue,
     citecolor=blue,      
     urlcolor=red,
}

% ----------------------------------------------------------------------------
% biblatex y hyperref
\DeclareCiteCommand{\cite}
  {\usebibmacro{prenote}}
  {\usebibmacro{citeindex}%
   \printtext[bibhyperref]{\usebibmacro{cite}}}
  {\multicitedelim}
  {\usebibmacro{postnote}}

\DeclareCiteCommand*{\cite}
  {\usebibmacro{prenote}}
  {\usebibmacro{citeindex}%
   \printtext[bibhyperref]{\usebibmacro{citeyear}}}
  {\multicitedelim}
  {\usebibmacro{postnote}}

\DeclareCiteCommand{\parencite}[\mkbibparens]
  {\usebibmacro{prenote}}
  {\usebibmacro{citeindex}%
    \printtext[bibhyperref]{\usebibmacro{cite}}}
  {\multicitedelim}
  {\usebibmacro{postnote}}

\DeclareCiteCommand*{\parencite}[\mkbibparens]
  {\usebibmacro{prenote}}
  {\usebibmacro{citeindex}%
    \printtext[bibhyperref]{\usebibmacro{citeyear}}}
  {\multicitedelim}
  {\usebibmacro{postnote}}

\DeclareCiteCommand{\footcite}[\mkbibfootnote]
  {\usebibmacro{prenote}}
  {\usebibmacro{citeindex}%
  \printtext[bibhyperref]{ \usebibmacro{cite}}}
  {\multicitedelim}
  {\usebibmacro{postnote}}

\DeclareCiteCommand{\footcitetext}[\mkbibfootnotetext]
  {\usebibmacro{prenote}}
  {\usebibmacro{citeindex}%
   \printtext[bibhyperref]{\usebibmacro{cite}}}
  {\multicitedelim}
  {\usebibmacro{postnote}}

\DeclareCiteCommand{\textcite}
  {\boolfalse{cbx:parens}}
  {\usebibmacro{citeindex}%
   \printtext[bibhyperref]{\usebibmacro{textcite}}}
  {\ifbool{cbx:parens}
     {\bibcloseparen\global\boolfalse{cbx:parens}}
     {}%
   \multicitedelim}
  {\usebibmacro{textcite:postnote}}

% ----------------------------------------------------------------------------
% Formato de texto
\setlength{\parindent}{0.5cm}  % sangria de 0.5cm
\linespread{1.5}  % interlineado de 1.5pt
% la alineacion de latex viene justificada por defecto

% ----------------------------------------------------------------------------
% Formato de títulos (titlesec)

% Este conjunto de lineas cambia el tamaño de cada título
% \titleformat{command}[shape]{format}{label}{sep}{before}{afer}
\titleformat{\chapter}{\normalfont\large\bfseries}{\thechapter.}{20pt}{\large}
\titleformat{\section}{\normalfont\large\bfseries}{\thesection}{12pt}{\large}
\titleformat{\subsection}{\normalfont\large\bfseries}{\thesubsection}{12pt}{\large}
\titleformat{\subsubsection}{\normalfont\large\bfseries}{\thesubsubsection}{12pt}{\large}

% Este conjunto de lineas reduce el espacio entre título y texto
% \titlespacing{command}{left}{beforesep}{aftersep}{right}
\titlespacing*{\chapter}{0pt}{12pt}{6pt}
\titlespacing*{\section}{0pt}{12pt}{6pt}
\titlespacing*{\subsection}{0pt}{12pt}{6pt}
\titlespacing*{\subsubsection}{0pt}{12pt}{6pt}


% ----------------------------------------------------------------------------
% Tikz para arboles de directorios
\tikzstyle{every node}
\tikzstyle{every node}=[draw=black,thick,anchor=west]
\tikzstyle{selected}=[draw=red,fill=red!30]
\tikzstyle{optional}=[dashed,fill=gray!50]

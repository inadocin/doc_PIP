Se ha obtenido un prediseño de los sistemas de admisión y escape que buscó
maximizar el rendimiento volumétrico y reducir la fracción de gases residuales
en un rango medio a medio alto de revoluciones del motor, obteniendo una curva
de rendimiento volumétrico con dos máximos locales a 2500 y 5000 RPM,
manteniendo valores de  $\eta_v$ mayores al 60\% hasta 8000 RPM.
%
Además, se obtuvo un mapa de $C_D$ que modeliza el funcionamiento de los
puertos con apertura de puerto y diferencia de presión como variables.

Junto con estos resultados se desarrolló un conjunto de \emph{scripts} que
permiten utilizar el simulador ICESym como una \emph{caja negra}, pudiendo
configurar, ejecutar y leer los resultados de una simulación, permitiendo
utilizar el simulador acoplado a otro programa.

Por otro lado, se pudieron comparar los resultados del simulador utilizando un valor
de $C_{D}$ constante en lugar de utilizar un mapa de $C_{D}$ en función de
valores de presión y apertura del puerto.

Como posible trabajo a futuro se propone continuar con el proceso de optimización
que se inició en este trabajo, realizando más iteraciones para refinar la
geometría obtenida.
%
Otro aspecto a trabajar a futuro es reducir la cantidad de gases residuales que
se producen por efecto del solape de cámaras.
%
Así como también explorar otras funciones objetivo para la optimización con el
algoritmo genético.

Se ha obtenido un prediseño de los sistemas de admisión y escape que buscó
maximizar el rendimiento volumétrico y reducir la fracción de gases residuales
en un rango medio a medio alto de revoluciones del motor, obteniendo una curva
de rendimiento volumétrico con una sintonía a 2000 RPM y manteniendo valores de
$\eta_v$ cercanos al 70\% para 6000 a 7000 RPM.
%
Además, se obtuvo un mapa de $C_D$ que modeliza el funcionamiento de los
puertos con apertura de puerto y diferencia de presión como variables.

Junto con estos resultados se desarrolló un conjunto de \emph{scripts} que
permiten utilizar el simulador ICESym como una \emph{caja negra}, pudiendo
configurar, ejecutar y leer los resultados de una simulación, esto permite
utilizar el simulador acoplado a otro programa.

Como trabajos a futuro se tiene la incorporación de los modelos de fricción
obtenidos en trabajos anteriores\parencite{roldan} a ICESym, la creación de un
optimizador genético híbrido, es decir, que permita utilizar algoritmos
genéticos para determinar los puntos más interesantes del dominio evaluado para
un problema dado y el uso de técnicas directas para encontrar el valor del
óptimo buscado.

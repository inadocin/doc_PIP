\section{Geometría y Ciclo Operativo del MRCVC}
%
En este apartado se describen algunos de los aspectos geométricos del motor y
ciclo operativo del MRCVC.

Los componentes principales del motor son: rotor, estator, paletas, bieletas,
rueda paralelizadoras, eje de motor, conducto de admisión y conducto de escape;
el motor analizado en este trabajo tiene 3 paletas con ápices agudos, que
corresponden a la geometría ideal del motor (ápices de paletas de radio nulo).
%
La forma de estos elementos se puede ver en la figura~\ref{fig:mrcvc} y en la
tabla~\ref{tab:geom_mrcvc} se resume el valor de los parámetros geométricos que
utilizados en este trabajo.

Uno de los aspectos más importantes de este motor es la geometría de la cámara
de combustión, su forma es tal que el volumen mínimo del ciclo permanece
constante por un período angular considerable, determinado por la geometría del
motor.
%
Este período es lo suficientemente grande para permitir que la combustión ocurra
casi en su totalidad a volumen constante, como se observa en la
figura~\ref{fig:mrcvc_vol_cte} y~\ref{fig:PV_mrcvc}.
%
Este tipo de combustión brinda una mejora en el rendimiento energético del motor
además, el balanceo de fuerzas que se obtiene por ser un motor rotativo permite
operar el motor a altas RPM y así alcanzar mayores potencias en comparación a
motores de tamaño o cilindrada similar.
%
Esta combinación de rendimiento y potencia que, en principio pueden ser
relativamente altos, hace atractivo el desarrollo de este motor.
%


\begin{figure}[ht]
  \centering
  \includegraphics[width=\textwidth]{gnuplot/vol.pdf}
  \caption{Variación del volúmen del MRCVC}\label{fig:mrcvc_vol_cte}
\end{figure}

\begin{figure}[ht]
  \centering
  \includegraphics[width=\textwidth]{gnuplot/vol_vs_pres.pdf}
  \caption{Ciclo operativo del MRCVC}\label{fig:PV_mrcvc}
\end{figure}


El ciclo operativo del MRCVC es un ciclo Otto en el que las carreras de
admisión, compresión, expansión y escape ocurren a medida que el fluido de
trabajo rota con respecto al eje del cigüeñal.
%
En la figura~\ref{fig:ciclo_mrcvc} se puede ver una progresiva del ciclo del
MRCVC con estas carreras representadas en azul para la admisión, compresión en
amarillo, expansión en rojo y escape o barrido en violeta.

Durante el ciclo se destaca un aspecto particular de este motor, siguiendo la
paleta de color negro se ve que durante el proceso de compresión y combustión,
las paletas que forman la frontera aguas arriba y aguas abajo de la cámara de
combustión cambian.
%
La paleta que delimita el frente de la cámara se retrasa con respecto a la
cámara con la que inició el ciclo, produciendo que este dure más de 1 revolución
resultando en aproximadamente $600^{\circ}$ de giro del cigüeñal.

Para un motor de $R=\lua{tex.print(myData.R)}$ mm y
$r=\lua{tex.print(myData.r)}$ mm  el volumen mínimo alcanzado permanece
constante por un período de $44.65^\circ$, como se puede ver en la
figura~\ref{fig:mrcvc_vol_cte} en donde se esquematiza la variación del volumen
con respecto al ciclo.

\begin{figure}[ht]
  \centering
  \includegraphics[width=\textwidth]{ciclo/ciclo_operativo.pdf}
  \caption{Ciclo operativo del MRCVC}\label{fig:ciclo_mrcvc}
\end{figure}

\begin{table}
    \centering
    \begin{tabular}{rcccccccc} \toprule
     Parámetro & n & R & r & $h_c$ & rc & V0 & $R_i$ & $R_e$ \\ \midrule
     Valor & \lua{tex.print(myData.n)} & \lua{tex.print(myData.R)} & \lua{tex.print(myData.r)} & \lua{tex.print(myData.hc)} & \lua{tex.print(myData.rc)} & \lua{tex.print(myData.V0)} & \lua{tex.print(trunc(myData.Ri))} & \lua{tex.print(trunc(myData.Re))} \\
     Unidades & --- & mm & mm & mm & --- & $cm^3$ & mm & mm \\ \bottomrule
    \end{tabular}
    \caption{Geometría del MRCVC}\label{tab:geom_mrcvc}
\end{table}

%%%%%%%%%%%%%%%%%%%%%%%%%%%%%%%%%%%%%%%%%%%%%%%%%%%%%%%%%%%%%%%%%%%%%%%%%%%%%%%

\subsection{Sistemas de intercambio de gases}
%
En un motor típico de combustión interna el sistema de intecambio de gases se
compone de una toma de aire, filtro, cuerpo de mariposa, puerto y conducto de
admisión, puerto y conducto de escape, catalizador y silenciador hasta
finalmente descargar en la atmósfera.

Para simplificar el sistema analizado no se tuvieron en cuenta elementos como:
mariposa, carburador, filtros de aire, convertidores catalíticos y demás; se
utilizó un sistema simplificado en el que solamente se tiene conducto de
admisión y escape junto con puertos de admisión y escape.
%
El eje de los conductos coincide con el eje del puerto, estos últimos hacen una
transición desde el diámetro del conducto hasta la altura de la ranura del
puerto en la cámara de combustión, en la
figura~\ref{fig:sistema_intercambio_gases} se esquematiza la geometría
mencionada.

\begin{figure}
    \centering
    \includegraphics[width=\textwidth]{ciclo/sistema_intercambio_gases.png}
    \caption{Esquema del sistema de intercambio de gases}\label{fig:sistema_intercambio_gases}
\end{figure}


En trabajos anteriores~\parencite{lopez13} se demostró que se tiene una mejor
\emph{performance} del motor si se ubican los puertos en el cuerpo central del
estator.
%
En dicho trabajo se realizó una optimización de la geometría mediante un barrido
paramétrico de las variables que determinan la forma, posición y reglaje de los
puertos, ya que es la ubicación angular de los puertos la que determina la
duración de los procesos de admisión y escape.
%
% Los ángulos que determinan el reglaje de los puertos de admisión y escape son
% los de apertura y de cierre del puerto.
%
Los puertos de admisión y escape están fijos en en la periferia del estator y su
posición se indica con los ángulos \emph{IIA}, e \emph{IFA} para la admisíon y
\emph{EIA} y \emph{EFA} para el escape.
%
En la figura~\ref{fig:angulos_escape} se indican estos para el puerto de escape.

\begin{figure}
    \centering
    \includegraphics[width=0.5\textwidth]{angulos_escape.png}
    \caption{Puerto de escape}\label{fig:angulos_escape}
\end{figure}

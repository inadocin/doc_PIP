El número de Courant es un indicador adimensional usado en CFD para evaluar el
tamaño de paso temporal necesario para una simulación en régimen transitorio o
para un tamaño de malla y velocidad determinados. Está relacioado con la
estabilidad de esquemas numéricos.

El número de Courant se define como
\begin{equation}
    C = \fraq{U \delta t}{\delta h}
\end{equation}

Dónde
\begin{description}
    \item [U] indica la velocidad de flujo
    \item [$\delta t$] indica el paso temporal
    \item [$\delta h$] es el tamaño característico de la malla
\end{description}

Para probelmas multi dimensionales se define como la suma de los componentes en
cada dimensión
\begin{equation}
    C = \sum_i \fraq{U_i \delta t}{\delta h_i}
\end{equation}

En términos de fluidodinámica computacional, el número de Courant indica que tanto
viaja la información (U) a través de una celda de la malla ($\delta h$), si $C_o > 1$
la información se propaga a través de una o más celdas por paso temporal, lo que
puede provocar que la simulación diverga o de resultados erroneos.

% Heywood 2Ed, pag 376
\section{Coeficiente de descarga ($C_D$)}

Se parte de la ecuación para flujo compresible a través de una restricción.

Para determinar el $C_D$ se debe conocer

\begin{description}
    \item[$p_0$] Presión de estancamiento antes de la restricción.
    \item[$T_0$] Temperatura de estancamiento anes de la restricción.
    \item[$p_T$] Presión estática justo después de la restricción.
    \item[$A_R$] Área de referencia.
    \item[$\dot{m}$] Caudal másico.
    \item[\gamma] Cociente de capacidades térmicas del gas.
\end{description}

Los valores de presión y temperatura se obtienen de los datos calculados con
ICESym. De estos valores se calcula el $\gamma$ del gas.

El caudal másico y la velocidad a las entradas y salidas se obtiene con
OpenFOAM.

Para flujo no bloqueado se utiliza:
$$ 
C_D^{-1} = \frac {A_R p_0} {\dot{m} R T_0^{1/2}}
           \left( \frac{p_T} {p_0} \right)^{1/\gamma}
           \left{ \frac{2\gamma}{\gamma-1}
           \left[1 - \left(\frac{p_T}{p_0}^(\gamma-1/\gamma)\right) \right] \right}^{1/2} 
$$

En caso del que el flujo esté bloqueado, es decir
$p_T/p_0 \le [2/\gamma+1)]^{\gamma/(\gamma - 1)}$
, la ecuación correspondiente es:

$$
C_D^{-1} =  \frac {A_R p_0} {\dot{m} (R T_0)^{1/2}}
            \gamma^{1/2}
            \left( \frac{2\gamma}{\gamma+1} \right)^{(\gamma+1)/(2(\gamma-1))}
$$

\subsection{Área de referencia}
El área de referencia utilizada por ICESym es el área de cortina.

$$ A_R = A_C = \pi D_v L_v $$

\subsection{Solape de cámaras}
Tanto al inicio como al cierre del puerto se ve solape de cámaras, por lo que
en estos intervalos angulares hay un valor de $C_D$ para cada cámara.


\chapter{Resultados}

En las tablas~\ref{tab:mapa_cd_admision} y~\ref{tab:mapa_cd_escape} se muestran los
resultados de realizar las flujometrías de los puertos de admisión y escape.

\begin{table}
  \centering
    \begin{tabular}{cccc} \toprule
      Caso  & lv        & $\Delta P$    & $C_{D}$   \\ \midrule
      0     & 0.016826  & -100331.39    &  0.213882 \\
      0     & 0.106775  & 5723.72       &  0.489375 \\
      0     & 0.016826  & -263797.72    &  0.011021 \\
      0     & 0.106775  & -3296.18      &  0.803197 \\
      0     & 0.016826  & -652902.78    &  0.011106 \\
      0     & 0.106775  & -9613.29      &  0.815804 \\
      0     & 0.016826  & -513568.73    &  0.011280 \\
      0     & 0.106775  & -3232.97      &  0.813186 \\
      1     & 0.026960  & -116996.12    &  0.375219 \\
      1     & 0.096641  & -3643.9       &  0.878414 \\
      1     & 0.026960  & -237724.11    &  0.018632 \\
      1     & 0.096641  & -6684.11      &  0.867774 \\
      1     &  0.02696  & -496509.46    &  0.111212 \\
      1     &  0.09664  & -18256.20     &  0.805830 \\
      1     & 0.026960  & -237724.11    &  0.022716 \\
      1     & 0.096641  & -6684.11      &  0.862647 \\
      2     & 0.047228  & -49343.47     &  0.541857 \\
      2     & 0.076373  & -5712.86      &  0.918061 \\
      2     & 0.047228  & -109348.67    &  0.487137 \\
      2     & 0.076373  & -17090.38     &  0.914182 \\
      3     & 0.067496  & 13.83         &  0.696967 \\
      3     & 0.071759  & -134.24       &  0.707263 \\
      3     & 0.067496  & -100073.52    &  0.731100 \\
      3     & 0.071759  & -24077.34     &  0.723965 \\
      4     & 0.075750  & -11793.31     &  0.946392 \\
      4     & 0.087764  & -33418.12     &  0.235717 \\
      4     & 0.087764  & -10715.70     &  0.221632 \\
      4     & 0.075750  & -5167.81      &  0.897169 \\
      6     & 0.123601  & -73.94        &  0.878522 \\ \bottomrule
    \end{tabular}
  \caption{Mapa de Cd del puerto de escape} \label{tab:mapa_cd_escape}
\end{table}

\begin{table}
  \centering
  \begin{tabular}{cccc} \toprule
      Caso  & lv        & $\Delta P$    & $C_{D}$   \\ \midrule
      0     & 0.014432  & -6574.97      &  0.206543 \\
      0     & 0.081937  & -87.24        &  0.828822 \\
      0     & 0.014432  & 21856.29      &  0.243975 \\
      0     & 0.081937  & -573.65       &  0.738459 \\
      0     & 0.014432  & -19738.67     &  0.222406 \\
      0     & 0.081937  & 519.60        &  0.487115 \\
      0     & 0.081937  & 1571.95       &  0.587277 \\
      0     & 0.014432  & 18077.97      &  0.256415 \\
      0     & 0.014432  & 2668.61       &  0.247292 \\
      0     & 0.081937  & 0.98          &  0.025970 \\
      2     & 0.062951  & -297.79       &  0.816487 \\
      2     & 0.081937  & 292.92        &  0.466147 \\
      2     & 0.062951  & -7374.88      &  0.980617 \\
      2     & 0.081937  & 4953.85       &  0.541619 \\
      3     & 0.071763  & 4092.13       &  0.501641 \\
      3     & 0.025832  & -3689.81      &  0.289852 \\
      4     & 0.069767  & -789.00       &  0.615690 \\
      4     & 0.069767  & 7869.92       &  0.599348 \\
      4     & 0.005564  & -12539.15     &  0.534555 \\
      4     & 0.005564  & -10091.84     &  0.583979 \\ \bottomrule
    \end{tabular}
  \caption{Mapa de Cd del puerto de Admisión} \label{tab:mapa_cd_admision}
\end{table}


La geometría obtenida luego de realizar la optimización con los mapas de Cd
incorporados a la simulación de ICESym se muestra en la figura \ref{fig:geom_nueva}.
%
Se puede ver que la geometría es similar a la inicial, siendo el puerto de
admisión algo menor en cuanto a diámetro que en el caso inicial.

Como es de esperarse, incorporar estos mapa al modelo del motor tiene un efecto
en el comportamiento del mismo, esto se puede observar principalmente en las
curvas de presión del motor.

\chapter{Algoritmo Genético}
\section{Introducción}
%%
%Se seleccionó un algoritmo genético para realizar la optimziación de la
%geometría del MRCVC por la simplicidad y facilidad de implementación.
%%
%Este método de optimización es de los llamados \emph{sin grediente} ya que no
%requieren información del gradiente de la función que se está estudiando.

%Los métodos que utilizan información sobre el gradiente de una función en un
%punto suelen ser muy eficientes si el espacio solución es diferenciable en todo
%el dominio, una desventaja de estos métdos es que pueden ser suceptibles a
%óptimos locales.
%%
%Por este motivo si la función a evaluar tiene muchos máximos/mínimos locales,
%el proceso de solución puede ser sensible al valor inicial utilizado.

%La relación entre el rendimiento volumétrico y los parámetros libres (diámetro
%y longitudes de tubos de admisión y escape, ángulos de apertura y cierre de los
%puertos) puede ser compleja con una variedad de óptimos locales,

%En casos en que no se puede asegurar que la función a estudiar sea
%diferenciable o que haya una gran cantidad de óptimos locales se vuelven
%atractivos los algoritmos genéticos.
%%
%Son básicamente métodos de búsqueda aleatoria que aprovechan la información de
%iteraciones previas para determinar la composición futura de la población.
%%
%Representan individuos como un conjunto de númreos que indican características
%particualres.
%%
%Se parte generalmente de una población o conjunto de individuos generados al
%azar, motivo por el cual estos algoritmos no son suceptibles a caer en una
%solución óptima a nivel local.

%Una ventaja importatnte de este método es que no es necesario hacer ninguna
%hipótesis sobre la diferenciabilidad o continuidad del dominio que se está
%estudiand, basta con poder evaluar la función para obtener un valor de aptitud.


%Para realizar la optimización de la geometría flujada se utiliza un algoritmo
%genético con una población de $N$ motores, los cuales se representan por una
%lista de $k$ números asociados a características geométricas de los sistemas de
%intercambio de gases, concreto cada motor se representa por:

%\begin{verbatim}
%[dta dte lta lte iia ifa eia efa]
%\end{verbatim}


%\subsection{Creación de la población}
%\subsection{Selección}
%\subsection{Cruza}
%\subsection{Mutación}
%\subsection{Nueva población}

%
Se seleccionó un algoritmo genético para realizar la optimización de la
geometría del MRCVC por la simplicidad y facilidad de implementación.
%
Si bien estos métodos no garantizan que se alcance un resultado óptimo, en la
práctica\cite{goldberg}\cite{shi} se ha observado que alcanzan soluciones muy
cercanas a las óptimas tras algunas iteraciones del método.
%
Este método de optimización no requiere información del gradiente de la función
que se está estudiando, esto es útil cuando no se puede asegurar la que existe
la derivada de la función en todo el dominio o cuando se tiene una función con
más de un máximo o mínimo local.
%

% Los métodos que utilizan información sobre el gradiente de una función en un
% punto suelen ser muy eficientes si el espacio solución es diferenciable en
% todo el dominio, una desventaja de estos métdos es que pueden ser suceptibles
% a óptimos locales.
%
% Por este motivo si la función a evaluar tiene muchos máximos/mínimos locales,
% el proceso de solución puede ser sensible al valor inicial utilizado.

% La relación entre el rendimiento volumétrico y los parámetros libres
% (diámetro y longitudes de tubos de admisión y escape, ángulos de apertura y
% cierre de los puertos) puede ser compleja con una variedad de óptimos
% locales,

En casos en que no se puede asegurar que la función a estudiar sea
diferenciable o que haya una gran cantidad de óptimos locales se vuelven
atractivos los algoritmos genéticos.
%
Son básicamente métodos de búsqueda aleatoria que aprovechan la información de
iteraciones previas para determinar la composición futura de la población,
representando individuos como un conjunto de números que indican
características particulares.
%
Se parte generalmente de una población o conjunto de individuos generados al
azar, motivo por el cual estos algoritmos no son susceptibles a caer en una
solución óptima a nivel local.


Es probable que no se alcance el óptimo, sin embargo se busca que la solución
encontrada sea la mejor de las evaluadas y que en la cantidad se obtenga una
solución satisfactoria.


¿Cómo difieren los AG de los métodos tradicionales de búsqueda?
%
\begin{enumerate}
    %
    \item "GA work with a coding of the parameter set, no the parameters
        themselves".
    %
    \item Trabajan con una población de puntos de datos, no con un solo punto.
    %
    \item Los AG funcionan con una función objetivo, no con derivadas u otra
        información.
    %
    \item Los AG usan reglas probabilísticas, no deterministas.
    %
\end{enumerate}

Los AG requieren que las variables del problema estén expresadas en forma de
coordenadas $(x_1, x_2, x_3, ..., x_n)$.

Otros métodos de optimización se mueven de un punto al siguiente en el espacio
solución, basándose en alguna regla de decisión, esto puede ser riesgo porque
se puede caer en un óptimo local.

Al buscar varios puntos por iteración, por lo que la probabilidad de caer en 
un óptimo local se ve reducida.


\section{Un algoritmo genético sencillo}
%
Un AG se puede construir con 3 operadores básicos:
\begin{enumerate}
    \item Reproducción
    \item Cruza
    \item Mutación
\end{enumerate}

La reproducción consiste en crear individuos a partir del puntaje que devuelve
en la función objetivo, la función objetivo es una medida del valor que
queremos optimizar.
%
Este paso significa que aquellos individuos que tengan valores, por ejemplo más
altos, de función objetivo tendrán más posibilidades de ser "copiados", de esta
forma se imita a la selección natural.

La cruza consiste en combinar los "vectores" de dos individuos para obtener uno
nuevo.

La mutación consiste en modificar aleatoriamente uno o más parámetros de cada
nuevo individuo.

Estos 3 simples mecanismos le dan a los AG su ¿poder?

La mutación juega un rol secundario pero muy importante, es secundario porque
se pueden alcanzar soluciones satisfactorias sin incluir este mecanismo, sin
embargo se utiliza con probabilidades pequeñas para evitar la pérdida temprana
de información relevante.
%
Si la probabilidad de mutación es muy alta, el AG se convierte en una simple
búsqueda aleatoria.

\section{\emph{Schema} y paralelismo implícito}

\emph{Schema} o esquema es una plantilla, un subconjunto de los parámetros que
definen al individuo en un AG que contienen información relevante al problema
en estudio.
%
Estas plantillas surgen de similitudes en posiciones y valores que dan buena
aptitud a un individuo.
%
Un AG procesa $n$ individuos y $n^k$ \emph{schematas} por iteración
\cite{goldberg}, cuanto mayor sea el tamaño del \emph{schema}, mayor la
probabilidad de ser truncado durante la cruza o mutación, esto produce que los
patrones de menor tamaño tengan mayor probabilidad de supervivencia de una
generación a otra.
%
Este proceso en el que bloques de información de menor tamaño al total de
parámetros que se está evaluado sobreviven de una generación a otra sin
necesidad de ingresar alguna modificación algoritmo es conocido como paralelismo
implícito.

\section{Implementación}
%
\subsection{Población}
%
El tamaño de la población se elige arbitrariamente en $N=100$ individuos, los
parámetros que representan a cada individuo son los que definen la geometría
del sistema de intercambio de gases, estos son:

\centerline{$(DTA,\ DTE,\ LTA,\ LTE,\ IIA,\ IFA,\ EIA,\ EFA)$}
%
Dónde:
%
\begin{itemize}
        %
    \item DTA es el diámetro de tubo de admisión
        %
    \item DTE es el diámetro de tubo de escape
        %
    \item LTA es el largo de tubo de admisión
        %
    \item LTE es el largo de tubo de escape
        %
    \item IIA es el ángulo de apertura del puerto de admisión
        %
    \item IFA es el ángulo de cierre del puerto de admisión
        %
    \item EIA es el ángulo de apertura del puerto de escape
        %
    \item EFA es el ángulo de cierre del puerto de escape
        %
\end{itemize}

Los diámetros pueden tomar valores de hasta 100mm, el largo de los
tubos puede variar entre $0.5$ y $1$ metro, los ángulos tanto de
admisión y escape se definen de la siguiente manera:

% aca un grafico de como están definidos los ángulos y el porque


Para seleccionar la población se utilizó 


\subsection{Reproducción}

Para crear la nueva población se debe elegir a los nuevos candidatos basándose
en los puntajes de candidatos previos, esto se implementó de la siguiente
manera:

\subsection{Cruza}
%
El operador de cruza se encarga de combinar los genes de dos individuos para
producir un individuo nuevo.

\subsection{Mutación}
%
La mutación juega un rol secundario pero importante, una pequeña probabilidad
de que alguno de los genes se modifique en un valor aleatorio contribuye a que
el algoritmo genético no se estanque en soluciones máximos o mínimos locales.

\subsection{Función objetivo}
%
La función objetivo es la encargada de dar puntaje a los individuos, en la
analogía con la selección natura, esta función es el ambiente, el que determina
que tan bien se desempeña un motor con respecto a otro en lo que respecta a
rendimiento volumétrico.
%
Inicialmente se propuso que la función objetivo sea la suma de los rendimientos
volumétricos a todas las velocidades simuladas, este tipo de funciones da como
resultado una curva de rendimiento volumétrico aserrada como se muestra en la
figura (falta).

Este comportamiento aserrado es poco deseable, por lo que se modificó la
función para conseguir una curva de rendimiento volumétrico suave.
%
Además se implementó una suma ponderada, para obtener un rendimiento volumétrico
máximo en un valor arbitrario de 5000 RPM.

Para prevenir el comportamiento aserrado se introdujo un sistema de penalidades,
que resta puntaje si la derivada de la función se modifica
de una velocidad a otra.


Otro tema a tener en cuenta es la convergencia prematura del algoritmo que es
causado por es la dominancia temprana de unos pocos individuos con puntaje
superior al resto de la población.
%
Para evitar esto se realiza un cambio de escala puntaje de todos los
individuos, alguno de los métodos utilizados son: transformación lineal,
truncado $\sigma$ y transformación exponencial.
%
El método seleccionado es la transformación lineal, que funciona bien \cite{goldberg}
si no se tienen valores de aptitud negativos.

\begin{equation}
    f' = af + b
\end{equation}

Los coeficientes $a$ y $b$ se eligen de modo tal de que haya unicidad entre el
puntaje sin modificar y el transformado, otra función es que se asigne un
puntaje de 2 a los individuos con mayor puntaje, lo que asegura que para
puntajes medios tengan posibilidades de generar un individuo como descendencia y
los mejores individuos tengan múltiples "hijos".

\chapter{Introducción}

Este trabajo presenta un procedimiento para optimizar la geometría del sistema
de intercambio de gases del \gls{mrcvc}
\cite{toth}, en particular de la geometría y posición en el cuerpo estatórico
de los puertos de admisión y escape, diámetro y longitud de los conductos
correspondientes.

El MRCVC es un proyecto nacido en la Universidad Nacional del Comahue en el
marco del \emph{Proyecto de Investigación Desarrollo de modelos y herramientas
para la simulación de problemas complejos en ingeniería mediante
fluidodinámica computacional (04/I-251)}. Actualmente se encuentra en etapa
de desarrollo.

La optimización de la geometría se realizó con un conjunto de herramientas, en
primer lugar para obtener las curvas de rendimiento volumétrico se utilizó un
simulador de motores de combustión interna, 

Se utilizó un simulador de motores de combustión interna para simular obtener
las curvas características del motor, curvas que se utilizan para dar un
puntaje a una algoritmo genético para optimizar el conjunto de parámetros que
definen la geometría de los sistemas de intercambio de gases y 

La motivación de este trabajo surge de continuar con el desarrollo del Motor
Rotativo de Combustión a Volumen Constante (MRCVC), en particular
mejorar el prediseño de los sistemas de intercambio de gases sentando las bases
para una futura optimización de los mismos en un motor con requisitos de diseño
establecidos.

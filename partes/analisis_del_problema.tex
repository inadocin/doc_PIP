% Heywood 2Ed, pag 376
\chapter{Análisis del problema}


\section{Optimización de la geometría}
\subsection{Simulación computacional del ciclo termodinámico MRCVC}
\subsubsection{ICESym}

ICESym \cite{icesym} es un simulador de motores de combustión interna
desarrollado en conjunto por la UNCo y UNL, utiliza modelos unidimensionales
para simular el flujo en los conductos de admisión y escape y modelos cero
dimensionales para el resto de los componentes.

El simulador incluye un modelo para el solape entre cámaras \cite{lopez16} y
modificaciones particulares al MRCVC, para este trabajo en particular  y con el
fin de obtener una mapa del coeficiente de descarga, se ha agregando la
diferencia de presión entre cámara y puerto como variable de modo que $Cd =
f(L_v, \delta P)$.

La combustión se realiza a volumen constante, por lo que es de esperarse
mayores rendimientos de conversión de combustible en relación a motores en los
que la combustión no se realiza a volumen constante.
% hace falta mostrar que esto es así? poner algún gráfico del heywood o cosas
% por el estilo. puedo citar el heywood?

El indicador que se tomará como referencia para evaluar y comparar diferentes
geometrías es el rendimiento volumétrico ($\eta_v$), este parámetro se define
como:

\begin{equation}
    \eta_v = \frac{m_a}{\rho_{a,i}V_d}
\end{equation}

Dónde:
\begin{description}
    \item[$m_i$] es la masa de mezcla fresca inductada
    \item[$\rho_{a,i}$] es la densidad del aire en el puerto de admisión
    \item[$V_d$] es el volumen desplazado
\end{description}

El rendimiento volumétrico tiene una dependencia compleja de varios factores,
este parámetro es el que da forma a las curvas de \emph{performance} que se
suelen ver en literatura ya que indica la cantidad de mezcla fresca disponible
para la combustión. En en caso de motores de inyección directa (tanto de CI SI)

\subsection{Algoritmo Genético}

\section{Flujometrías}
\subsection{}
\subsection{Coeficiente de descarga ($C_D$)}

Se parte de la ecuación para flujo compresible a través de una restricción.

Para determinar el $C_D$ se debe conocer

\begin{description}
    \item[$p_0$] Presión de estancamiento antes de la restricción.
    \item[$T_0$] Temperatura de estancamiento antes de la restricción.
    \item[$p_T$] Presión estática justo después de la restricción.
    \item[$A_R$] Área de referencia.
    \item[$\dot{m}$] Caudal másico.
    \item[$\gamma$] Cociente de capacidades térmicas del gas.
\end{description}

Los valores de presión y temperatura se obtienen de los datos calculados con
ICESym. De estos valores se calcula el $\gamma$ del gas.

El caudal másico y la velocidad a las entradas y salidas se obtiene con
OpenFOAM.

Para flujo no bloqueado se utiliza:
% \begin{math}
% C_D^{-1} = \frac {A_R p_0} {\dot{m} R T_0^{1/2}}
%            \left( \frac{p_T} {p_0} \right)^{1/\gamma}
%            \left{ \frac{2\gamma}{\gamma-1}
%            \left[1 - \left(\frac{p_T}{p_0}^(\gamma-1/\gamma)\right) \right] \right}^{1/2} 
% \end{math}

En caso del que el flujo esté bloqueado, es decir
$p_T/p_0 \le [2/\gamma+1)]^{\gamma/(\gamma - 1)}$
, la ecuación correspondiente es:

\begin{math}
C_D^{-1} =  \frac {A_R p_0} {\dot{m} (R T_0)^{1/2}}
            \gamma^{1/2}
            \left( \frac{2\gamma}{\gamma+1} \right)^{(\gamma+1)/(2(\gamma-1))}
\end{math}

\subsection{Área de referencia}
El área de referencia utilizada por ICESym es el área de cortina.

$$ A_R = A_C = \pi D_v L_v $$

\subsection{Solape de cámaras}
Tanto al inicio como al cierre del puerto se ve solape de cámaras, por lo que
en estos intervalos angulares hay un valor de $C_D$ para cada cámara.

\section{Criterios adoptados}
\subsection{Geometría}
El puerto se hace recto, igual se podría hacer una entrada más suave.

La altura de la ranura se adopta en 2/3 del alto de la cámara, siendo $h_c=0.0441\ mm$

El eje del puerto se hace perpendicular a una línea que pasa entre el centro
del motor y el la línea media del puerto.


% Capítulo 3 - Metodología:

% En este capítulo describo como es el procedimiento realizado en cada paso de
% la optimización:

% -> optimización algoritmo genético y simulación con icesym, tendría que
%    explicar como funciona icesym y el optimizador
% -> freecad + salome
% -> openfoam

\section{Metodología}

Se simulará un MRCVC de 3 paletas usando el simulador de motores de combustión
interna ICESym \cite{icesym} con el propósito de obtener curvas de rendimiento
volumétrico.
%
Estas serán el principal criterio para evaluar el diseño de los sistemas de
intercambio de gases, que consisten de los puertos y conductos de admisión y
escape.
%
La geometría se optimizará mediante un algoritmo genético, el cual transforma
los datos de rendimiento volumétrico en un puntaje representativo de cada
motor.


La simulación con ICESym requiere del conocimiento previo de los coeficientes
de descarga de los puertos los cuales inicialmente serán estimados para poder
realizar la primer iteración con el simulador.
%
Se utilizará la geometría obtenida en esta primer iteración para crear modelos
en CAD de los puertos de admisión y escape para realizar flujometrías virtuales
y así poder obtener los coeficientes de descarga de cada puerto en distintos
grados de apertura.


\section{ICESym}



\section{Algoritmo genético}

Para realizar la optimización de la geometría flujada se utiliza un algoritmo
genético con una población de $N$ motores, los cuales se representan por una
lista de $k$ números asociados a características geométricas de los sistemas de
intercambio de gases, concreto cada motor se representa por:

\begin{verbatim}
[dta dte lta lte iia ifa eia efa]
\end{verbatim}


\section{CAD}
El modelo 3D de los puertos de admisión, escape y los componentes internos del
motor que afectan el flujo y son relevantes a la flujometría se realizaron con
FreeCAD\cite{freecad} para generar un archivo en formato $.BREP$ y luego
salome\cite{salome} para obtener un archivo .stl "cerrado" en formato ASCII

Dada la cantidad de geometrías posibles y el tiempo que toma el proceso, se
realizaron algunas simplificaciones


\section{Flujometrías}

\subsection{Estudio de convergencia de malla}

El tamaño de la malla debe ser lo suficientemente pequeño para que el resultado
no depende del tamaño de la malla, es decir que el resultado no cambie si se
usan tamaños de celdas menores.
%
Dada la cantidad de posiciones diferentes a ensayar y la gran cantidad de
tiempo que significa realizar el estudio para cada posición, se tomará el
tamaño mínimo de malla como el tamaño que surga de realizar el estudio de
convergencia de malla en dos posiciones del rotor, una posición cerca de la
apertura del puerto y otra cercana al cierre del mismo.
%
Se eligen estas posiciones porque son las que presentan los mayores gradientes
de presión.


El procedimiento a seguir es el siguiente:
% en lugar de una lista numerada podría hacer un algoritmo
\begin{enumerate}
    \item surfaceFeatures
    \item blockMesh
    \item snappyHexMesh
    \item checkMesh
\end{enumerate}


\subsection{Condiciones de contorno}
Para determinar las condiciones de contorno se usan los datos 
\subsubsection{Con solape de cámaras}
\subsubsection{Sin solape de cámaras}

\subsection{Modelos de viscosidad}
\subsubsection{k-epsilon}
\subsubsection{k-omega}

\subsection{Tipos de flujo}
\subsubsection{Flujo compresible}
\subsubsection{Flujo incompresible}

\section{Retroalimentación}


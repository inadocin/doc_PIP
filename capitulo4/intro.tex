Los resultados obtenidos en cada uno de los pasos de este trabajo se detallan en
este capítulo, comenzando por el motor obtenido en la primer iteración de
optimización con el algoritmo genético junto con el modelo de CAD generado.
%
Luego se muestran los resultados de las flujometrías realizadas a partir del
modelo de CAD, incluyendo las mallas obtenidas para algunos casos seleccionados
y el resultado detallado de algunas de las flujometrías, finalizando con el mapa
de $C_{D}$ obtenido, tanto para el puerto de admisión como para
el puerto de escape.

Por último se presentan los resultados de la segunda ronda de optimización con
el algoritmo genético, en la que se utilizó el mapa de $C_{D}$ obtenido en el
paso previo.
%
% En esta sección se muestra además el modelo de CAD generado para esta
% geometría.

% Se simuló un MRCVC de 3 paletas con ICESym~\parencite{icesym} con el propósito
% de obtener curvas de rendimiento volumétrico.
% %
% Estas serán el principal criterio para evaluar el diseño de los sistemas de
% intercambio de gases, que consisten de los puertos y conductos.
% %
% La geometría se optimizará mediante un algoritmo genético, el cual transforma
% los datos de rendimiento volumétrico en un puntaje representativo de cada
% motor.


% La simulación con ICESym requiere del conocimiento previo de los coeficientes
% de descarga de los puertos, en la primer iteración se asume un valor constante
% para cada puerto que vale 0.7 y 0.75 para admisión y escape respectivamente.
% %
% Se utilizará la geometría obtenida en esta primer iteración para crear modelos
% en CAD de los puertos de admisión y escape, junto con una porción de la cámara
% de combustión para realizar flujometrías virtuales.
% %
% De este modo se puede obtener los coeficientes de descarga de cada puerto en
% distintos grados de apertura.

% Con el resultado de las flujometrías se construye un mapa del $C_D$ en función
% de la alzada, que representa la posición del puerto y la diferencia de presión
% entre el puerto y la cámara de combustión.

% Este mapa se introduce en ICESym y se vuelve a realizar el proceso de
% optimización con el algoritmo genético.

% El Simulador de Motores de Combustión Interna (ICESym) permite la simulación de
% motores tanto alternativos como rotativos en general y el MRCVC en particular,
% en su código se incluyen modelos de la geometría del motor, la transferencia de
% calor y el solape de cámaras.

% %%%%%%%%%%%%%%%%%%%%%%%%%%%%%%%%%%%%%%%%%%%%%%%%%%%%%%%%%%%%%%%%%%%%%%%%%%%%%%%


% %%%%%%%%%%%%%%%%%%%%%%%%%%%%%%%%%%%%%%%%%%%%%%%%%%%%%%%%%%%%%%%%%%%%%%%%%%%%%%%

% \section{CAD}
% %
% El modelo 3D de los puertos de admisión, escape y los componentes internos del
% motor que afectan el flujo y son relevantes a la flujometría se realizaron con
% FreeCAD\parencite{freecad} para generar un archivo en formato \emph{.BREP} para cada
% posición analizada.
% %
% Estos archivos son importados al software salome\parencite{salome}, ese permite
% generar una malla cerrada que se puede utilizar para OpenFOAM.\@
% %
% Antes de generar la malla se aplican los nombres de los parches, en los que
% luego se toman de referencia para aplicar las reglas para realizar el
% refinamiento con snappyHexMesh, aplicar condiciones de contorno, iniciales y
% demás.
% %
% Se genera la malla con el mallador NETGEN 1D-2D-3D y se exporta parche por
% parche en formato \emph{ASCII.stl}.

% %
% El archivo \emph{stl} se utiliza en OpenFOAM para generar la malla con
% \emph{snappyHexMesh}.

% Dada la cantidad de geometrías posibles y el tiempo que toma el proceso, se
% realizaron algunas simplificaciones, las cuales se listan a continuación.

% %%%%%%%%%%%%%%%%%%%%%%%%%%%%%%%%%%%%%%%%%%%%%%%%%%%%%%%%%%%%%%%%%%%%%%%%%%%%%%%

% % \subsection{Geometría}
% %
% \begin{enumerate}

%     \item La interfaz entre los puertos de admisión y escape con sus respectivos
%         conductos de conexión con la atmósfera es de sección circular.
%         %
%     \item La altura de la ranura se adopta en $2/3$ del alto de la cámara,
%         siendo $h_c=\lua{tex.print(myData.hc)}\ mm$.
%         %
%     \item El eje del puerto se hace perpendicular a una línea que pasa entre el
%         centro del motor y el la línea media del puerto.
% \end{enumerate}


% %%%%%%%%%%%%%%%%%%%%%%%%%%%%%%%%%%%%%%%%%%%%%%%%%%%%%%%%%%%%%%%%%%%%%%%%%%%%%%%

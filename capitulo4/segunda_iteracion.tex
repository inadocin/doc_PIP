\section{Segunda Iteración y Resultado Final}
%
En la segunda iteración se utilizó el mapa de $C_D$ para la admisión y escape
como dato de entrada de ICESym.
%
Con esto se realizó una serie de corridas de optimización con el algoritmo
genético, de las cuales se seleccionaron los mejores candidatos.
%
Se obtuvieron 3 candidatos principales, indicados como \emph{$run_{34}$},
\emph{$run_{38}$}, \emph{$run_{51}$} cuyas geometrías se indican en la
Tabla~\ref{tab:2iter_geom}.
%
Se ve que los diámetros son similares y que la mayor variación se da en los
largos de los conductos de admisión y escape.
%
% Exceptuando la corrida $run_{51}$ los ángulos de apertura de los puertos de
% admisión se mantienen cercanos Figura \ref{fig:2iter_general}.
%
Para determinar cuál de todos es el más prometedor, se compararon las curvas de
rendimiento volumétrico, torque, potencia y fracción de gases residuales, las
cuales se muestran en la Figura~\ref{fig:comparativa_segunda_iter}.

\begin{table}
  \centering
  \begin{tabular}{ccccccccc}\toprule
    Corrida   & DTA   & DTE   & LIT   & LET   & IIA   & IFA    & EIA    & EFA \\
    -         & [mm]  & [mm]  & [m]   & [m]   & [gra] & [gra]  & [gra]  & [gra] \\ \midrule
    run 03-04 & 0,099 & 0,063 & 0,839 & 0,79  & 3,226 & 60,0   & 56,613 & 36,29 \\
    run 05-04 & 0,09  & 0,095 & 0,742 & 0,597 & 8,387 & 60,0   & 68,226 & 23,226 \\
    run 09-04 & 0,082 & 0,091 & 0,79  & 1,758 & 3,226 & 60,968 & 75,484 & 39,194 \\
    run 10-04 & 0,096 & 0,06  & 1,129 & 1,468 & 6,129 & 66,774 & 62,419 & 40,645 \\ \bottomrule
  \end{tabular}
  \caption{Geometrías de segunda iteración}\label{tab:2iter_geom}
\end{table}

\begin{figure}
  \centering
  \includegraphics{gnuplot/comparativa_segunda_iter_mep.pdf}
  \caption{Comparativa candidatos} \label{fig:comparativa_segunda_iter}
\end{figure}

De las figuras comparadas se selecciona la corrida denominada como ``run 3-4''
porque a pesar de tener una curva de rendimiento volumétrico suave con máximo en
4000 RPM y no a

% \begin{figure}[ht]
%   \centering
%   \begin{subfigure}[b]{.5\textwidth}
%     \centering
%     \includegraphics{gnuplot/segunda_iter_pot.pdf}
%     \caption{Potencia indicada y al freno} \label{fig:primer_op}
%   \end{subfigure}%
%   \begin{subfigure}[b]{.5\textwidth}
%     \centering
%     \includegraphics{gnuplot/segundo_rend_vol.pdf}
%     \caption{Rendimiento Volumétrico}
%   \end{subfigure}
%     \caption{Segunda Iteración} \label{fig:primer_op}
% \end{figure}


El motor tiene una potencia máxima de 117 CV a las 6500 RPM y un par máximo de
177Nm a 4000 RPM.
%
Este coincide con el máximo de rendimiento volumétrico de $\sim 0,845$.
%
En la Figura~\ref{fig:PoTi_segunda_op} se notan los efectos del coeficiente de
descarga en la simulación del motor, comparando los resultados de la primera
iteración con coeficientes de descarga constantes a la segunda con el mapa de
$C_{D}$ en función de la presión y grado de apertura del puerto.

\begin{figure}[ht]
  \centering
  \includegraphics{gnuplot/comparativa.pdf}
  \caption{Comparativa de Torque y potencia al freno} \label{fig:PoTi_segunda_op}
\end{figure}

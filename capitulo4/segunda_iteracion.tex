\section{Segunda iteración y resultado final}
%
En la segunda iteración se utilizó el mapa de $C_D$ para la admisión y escape
como dato de entrada para ICESym, con esto se realizó una serie de corridas de
optimización con el algoritmo genético, de las cuales se seleccionaron los
mejores candidatos.
%
Se obtuvieron 3 candidatos principales, indicados como \emph{$run_34$},
\emph{$run_38$}, \emph{$run_51$}; cuyas curvas de rendimiento volumétrico y
fracción de gases residuales se indican en la figura \ref{fig:2iter_general}.
%
Para determinar cuál de todos es el más promoetedor, se compararon las curvas
de presión, torque y potencia, las cuales se muestran en las figuras
\ref{fig:2iter_presion}, \ref{fig:2iter_torque} y \ref{fig:2iter_potencia}
respectivamente.

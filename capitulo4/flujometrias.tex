%%%%%%%%%%%%%%%%%%%%%%%%%%%%%%%%%%%%%%%%%%%%%%%%%%%%%%%%%%%%%%%%%%%%%%%%%%%%%%%
\section{Flujometrías}

De la pirmer iteración se obtuvo la geometría y datos operativos del motor, con
esto se representó la curva de diferencia de presión ($\Delta P$) en función de
la alzada ($l_{v}$) de ambos puertos para diferentes velocidades de giro.
%
Con estas gráficas se identificó puntos de mayor interés en los cuales realizar
las flujometrías, los pares $(l_{v}, \Delta P)$ seleccionados para modelar el
flujo del puerto se detallan en las figuras~\ref{fig:delta_p_admision}
y~\ref{fig:delta_p_escape}.
%
Inicialmente se propusieron 51 flujometŕias pudiendo realizar un total de 36
simulaciones que devoliveron 56 valores de $C_{D}$.

\begin{figure}[h]
  \centering
  \includegraphics[width=\textwidth]{flujometrias/admision_delta_p_anot.png}
  \caption{Flujometrías puerto de admisión}\label{fig:delta_p_admision}
\end{figure}

\begin{figure}[h]
  \centering
  \includegraphics[width=\textwidth]{flujometrias/escape_delta_p_anot.png}
  \caption{Flujometrías puerto de escape}\label{fig:delta_p_escape}
\end{figure}


% \begin{table}[ht!]
%   \centering
%   \begin{tabular}{cccccc}\toprule
%     GRA/RPM & 1000 & 2000 & 3000 & 7000 & 9000 \\ \midrule
%     0       & -    & X    & -    & X    & - \\
%     10      & -    & -    & -    & X    & - \\
%     50      & -    & -    & X    & X    & X \\
%     100     & X    & -    & -    & X    & - \\
%     590     & X    & X    & X    & -    & - \\ \bottomrule
%   \end{tabular}
%   \caption{Flujometrías puerto de admisión}\label{tab:flujometrias_admision}
% \end{table}


% \begin{table}[ht!]
%   \centering
%   \begin{tabular}{cccccccc}\toprule
%     GRA/RPM & 1000 & 2000 & 3000 & 4000 & 7000 & 8000 & 9000 \\ \midrule
%     405     & X    & X    & -    & -    & X    & -    & X \\
%     410     & -    & X    & -    & X    & X    & X    & - \\
%     420     & -    & -    & -    & X    & -    & X    & - \\
%     430     & -    & X    & -    & -    & -    & -    & - \\
%     440     & X    & -    & -    & -    & X    & -    & X \\
%     500     & -    & -    & X    & -    & X    & X    & - \\ \bottomrule
%   \end{tabular}
%   \caption{Flujometrías puerto de escape}\label{tab:flujometrias_escape}
% \end{table}

Algunas flujometrías se realizaron en tres etapas, partiendo de una malla gruesa
con celdas de 15mm de tamaño inicial, culminando en celdas de 5mm.
%
En otros se realizó directamente la flujometría con mallas base de 5mm.

\parbox{\textwidth}{\colorbox{yellow}{NOTA: falta agregar una tabla o gráfica
con estas flujometrías}}

En general se simuló alrededor de 0.02 segundos de flujo  hasta alcanzar un
valor estable del caudal másico, como se indica en la
figura~\ref{fig:adm_10_7000rpm} donde se muestra el desarrollo de la simulación
en términos de $\dot{m}$ para el puerto de admisión con el cigüeñal en
$\theta=10^{\circ}$.
%
La línea anaranjada sobre el final de la simulación representa la porcióporción
de datos que se seleccionó para calcular $\dot{m}$, esto se realizó tomando la
media de los últimos $n$ valores obtenidos, los resultados de las flujometŕias
para puerto de admisión y escape se presentan en el Anexo~\ref{anexo:1}.
%
La totalidad de puntos evaluados se presentan en las
tablas~\ref{tab:mapa_cd_admision} y~\ref{tab:mapa_cd_escape} para el puerto de
admisión y escape respectivamente.

\begin{figure}[ht]
  \centering
  \includegraphics[width=\textwidth]{flujometrias/admision_SFV_10_0.png}
  \caption{Puerto de admisión 10º \@ 7000RPM}\label{fig:adm_10_7000rpm}
\end{figure}

Como se mencionó en el apartado~\ref{capitulo:DESARROLLO}, la modificaión
realizada a ICESym para funcionar con un mapa de $C_{D}$ dependiente de dos
variables requiere que los datos de entrada estén distribuidos en una grilla
rectangular.
%
Los datos obtenidos no forman una grilla rectangular, se utilizó un método de
interpolación de punto más cercano suavizado por promedio móvil con $S=2$ para
generar dicha grilla a partir de los puntos conocidos de $C_{D}$, el resultado
se ve en las figuras~\ref{fig:mapa_cd_admision} y~\ref{fig:mapa_cd_escape}.


\subsection{Puerto de Admisión}
%
En el mapa del coeficiente de descarga para el puerto de admisión se puede
observar en rojo las zonas de menor eficiencia del escurrimiento. Esto ocurre
para valores chicos de alzada, relacionados con la apertura y cierre del puerto
donde las presiones y velocidades de flujo involucradas son mayores aumentando
las pérdidas de carga.

\begin{figure}[ht!]
    \centering
    \includegraphics[width=\textwidth]{mapa_cd/mc_s2_mapa_adm.png}
    \caption{Puerto de admisión}\label{fig:mapa_cd_admision}
\end{figure}

\subsubsection{$C_{D}$ máximo}
%
En el mapa del puerto de admisión se observa un máximo de $C_{D,\max}\simeq 0.6$
para para $l_{v}=62.95 mm$ y $\Delta P\simeq -7.37 KPa$, obteneniendo un flujo
hacia afuera del puerto de $122.09 g/seg$, para este caso se ve un reflujo de
gases residuales apenas abre el puerto de admisión, este caso corresponde a un
ángulo de cigüeñal de $10^{\circ}$ a 7000 RPM.

La flujometría correspondiente al último instante de la simulación se muetra en
la figura~\ref{fig:adm_cd_max}, las líneas de corrienteven  están coloreadas
según el módulo de la velocidad y las flechas indican el sentido de flujo.
%
La mayor velocidad de flujo se da en el gas que sale de la cámara de combustión,
este corresponde a masa residual que qeuda atrapado luego del barrido del puerto
de escape.

\begin{figure}[ht!]
    \centering
    \includegraphics[width=\textwidth]{flujometrias/adm_cd_max.png}
    \caption{Admisión - Valor máximo de $C_{D}$}\label{fig:adm_cd_max}
\end{figure}

\subsubsection{$C_{D}$ mínimo}
%
El menor valor de $C_{D}$ se obtiene para el puerto de admisión en una
posición muy próxima a la apertura, $C_{D,\min}\simeq 0.12$ con un flujo hacia
el puerto de $\dot{m}\simeq 5 g/seg$, $l_{v}=144.3 mm$ y $\Delta P=-6.57 KPa$ para
el puerto a $590^{\circ}$ y 1 000 RPM, ver figura~\ref{fig:adm_cd_min}.

\begin{figure}[ht!]
    \centering
    \includegraphics[width=\textwidth]{flujometrias/adm_cd_min.png}
    \caption{Admisión - Valor mínimo de $C_{D}$}\label{fig:adm_cd_min}
\end{figure}

\subsubsection{$\dot{m}$ máximo}
%
En términos de flujo másico, el máximo es $\dot{m}_{\max}\simeq 70 g/seg$ y ocurre
durante un período de máxima apertura del puerto con $l_{v}=81.94 mm$ y
$\Delta P=4.95 KPa$ siendo $C_{D}=0.32874$, ver figura~\ref{fig:adm_m_max}.

\begin{figure}[ht!]
    \centering
    \includegraphics[width=\textwidth]{flujometrias/adm_cd_max.png}
    \caption{Admisión - Valor máximo de $\dot{m}$}\label{fig:adm_m_max}
\end{figure}

\subsection{Puerto de Escape}
%
En la figura~\ref{fig:mapa_cd_escape} se ilustra el mapa obtenido para el puerto
de escape, la tendencia es similar al puerto de admisión con la diferencia que
no hay gradientes positivos de presión, la zona con mayores valores de $C_{D}$
corresponde a la máxima apertura del puerto y valores de presión intermedios.
%

\begin{figure}[ht!]
    \centering
    \includegraphics[width=\textwidth]{mapa_cd/mc_s2_mapa_esc.png}
    \caption{Puerto de escape}\label{fig:mapa_cd_escape}
\end{figure}

\subsubsection{$C_{D}$ máximo}
%
Para el puerto de escape se observa el máximo $C_{D,\max}=0.57686$ para
$l_{v}=87.76 mm$, $\Delta_{P}=-1 KPa$ con un flujo másico de $145 g/s$ hacia
afuera para $440^{\circ}$ a 9000 RPM, ver figura~\ref{fig:esc_cd_max}.

\begin{figure}[ht]
    \centering
    \includegraphics[width=\textwidth]{flujometrias/esc_cd_max.png}
    \caption{Escape - Valor máximo de $C_{D}$}\label{fig:esc_cd_max}
\end{figure}

\subsubsection{$C_{D}$ mínimo}
%
Para $l_{v}=16.83 mm$ y $\Delta_{P}=-652.9 KPa$ se obtiene $C_{D, \min}=0.09631$
donde el flujo se encuentra bloqueado por la alta diferencia de presiones,
alcanzando $\dot{m}=38.6 g/seg$, ver figura~\ref{fig:esc_cd_min}.

\begin{figure}[ht]
    \centering
    \includegraphics[width=\textwidth]{flujometrias/esc_cd_max.png}
    \caption{Escape - (CAMBIAR POR CD MIN) Valor mínimo de $C_{D}$}\label{fig:esc_cd_min}
\end{figure}

\subsubsection{$\dot{m}$ máximo}
%
El flujo másico máximo es $\dot{m}=176.1 g/seg$ para $l_{v}=87.76 mm$ y
$\Delta_{P}=-334 KPa$, esto con el ciclo a $440^{\circ}$ y 7000 RPM,
ver figura~\ref{fig:esc_m_max}.

\begin{figure}[ht]
    \centering
    \includegraphics[width=\textwidth]{flujometrias/esc_m_max.png}
    \caption{Escape - Valor máximo de $\dot{m}$}\label{fig:esc_m_max}
\end{figure}


% NOTA: esto debería ir en la parte de la segunda iteración
% La geometría obtenida luego de realizar la optimización con los mapas de $C_D$
% incorporados a la simulación de ICESym se muestra en la figura~\ref{fig:geom_nueva}.
% %
% Se puede ver que la geometría es similar a la inicial, siendo el puerto de
% admisión algo menor en cuanto a diámetro que en el caso inicial.
% %
% Como es de esperarse, incorporar estos mapa al modelo del motor tiene un efecto
% en el comportamiento del mismo, esto se puede observar principalmente en las
% curvas de presión del motor.

% Para obtener el mapa se tomaran valores de flujo másico en las combinaciones de
% $(\Delta P, l_v)$ que están indicadas en la tabla~\ref{tab:casos}.
% %
% Como se ve en la Figura~\ref{fig:flujometrias}, los puntos a evaluar son los
% listados en la Tabla~\ref{tab:casos}.
% %
% El mapa de $C_D$ obetnido a partir de las flujoemtrías se lista en la
% tabla~\ref{tab:mapaAdm} y~\ref{tab:mapaEsc} para los mapas de admisión y escape
% respectivamente.
%
% En la figura \ref{fig:flujometrias} se ve que se eligieron más cantidad de
% muestreos en las zonas donde hay mayores cambios de presión.
%
% La figura \ref{fig:flujometrias} fué obtenida a partir de los resultados del
% simulador ICESym, restando para las velocidades seleccionadas la presión de la
% cámara a la presión en la boca del puerto.

% \begin{figure}
%     \centering
%     \includegraphics[width=1\textwidth]{flujometrias_admision.png}
%     \caption{Flujometrías para el puerto de Admisión}\label{fig:flujometrias}
% \end{figure}
